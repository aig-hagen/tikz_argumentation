\documentclass{scrartcl}

\usepackage[top=1.35in, bottom=1.33in, left=1.35in, right=1.35in]{geometry}
\usepackage[macros,beamer]{argumentation}       % Main Package
\usepackage[T1]{fontenc}                        % Font Encoding
\usepackage[utf8]{inputenc}                     % Input font encoding
\usepackage[english]{babel}                     % Language
\usepackage{subcaption}                         % Multi-part Figures
\usepackage[nohyperref]{doc}                    % Showing Commands (must load before tcolorbox)
\usepackage[breakable]{tcolorbox}               % Fancier Examples
\usepackage{booktabs}                           % Fancier Tables
\usepackage{enumitem}                           % Fancier Enumeration
\usepackage[hidelinks,                          % Setting Meta-Data and hyperlinks
            pdfauthor={Lars Bengel},
            pdftitle={The argumentation LaTeX-package},
            pdfsubject={Documentation of the argumentation LaTeX-package},
            pdfkeywords={argumentation,tikz,graphs}]{hyperref}

%%% Document settings
\KOMAoptions{
    paper=a4,
    fontsize=10pt,
    cleardoublepage=empty,
    footinclude=true
}

%%% Various commands used in the document
\newcommand{\todo}[1]{\textcolor{magenta}{TODO: #1}} % for todos
\newcommand{\tikzname}{Ti\emph{k}Z\xspace}
\newcommand{\argumentation}{\texttt{argumentation}\xspace}
\newcommand{\opt}[2][red]{\textcolor{#1}{\ttfamily \meta{#2}}}
\newtheorem{example}{Example}
\tcolorboxenvironment{example}{colframe=aigblue,colback=aigyellow!30,breakable,before skip=10pt,after skip=10pt}

%%% Taken from ltxdoc-class
\def\cmd#1{\cs{\expandafter\cmd@to@cs\string#1}}
\def\cmd@to@cs#1#2{\char\number`#2\relax}
\DeclareRobustCommand\cs[2][-1.6cm]{\hspace{#1}\texttt{\char`\\#2}}

\title{The \argumentation Package}
\author{Lars Bengel\footnote{Please report any issues at \url{https://github.com/aig-hagen/tikz_argumentation}}\\\small lars.bengel@fernuni-hagen.de}
\date{Version 1.4 [2024/10/31]}

\begin{document}
\maketitle

\begin{minipage}{.26\textwidth}
    \centering
        \begin{af}[argumentstyle=gray,namestyle=monospace]
            \argument{a}
            \argument[right=of a1]{b}
            \argument[below=of a1]{c}
            \argument[right=of a3]{d}
    
            \attack{a1}{a2}
            \attack{a2}{a3}
            \attack[bend right]{a3}{a4}
            \label{af:ex1}
        \end{af}
\end{minipage}
\begin{minipage}{.5\textwidth}
    \begin{small}
    \begin{verbatim}
        \begin{af}[argumentstyle=gray,namestyle=monospace]
            \argument{a}
            \argument[right=of a1]{b}
            \argument[below=of a1]{c}
            \argument[right=of a3]{d}
    
            \attack{a1}{a2}
            \attack{a2}{a3}
            \attack[bend right]{a3}{a4}
            \label{af:ex1}
        \end{af}
    \end{verbatim}
    \end{small}
\end{minipage}


\tableofcontents
\newpage

%\sffamily
%%%%%%%%%%%%%%%%%%%%%%%%%%%%%%%%%%%%%%%%%%%%%%%%%%%%%%%%%%%%%%%%%%%%%%%%%%%%%%%%%%%%%%%%%%%%%%%%%%%%%%%%%%%
\section{Quick Guide}\label{sec:quick}
%%%%%%%%%%%%%%%%%%%%%%%%%%%%%%%%%%%%%%%%%%%%%%%%%%%%%%%%%%%%%%%%%%%%%%%%%%%%%%%%%%%%%%%%%%%%%%%%%%%%%%%%%%%
\noindent
To create an argumentation framework in your \LaTeX-document, you first have to import the \argumentation package in the preamble:\\

\noindent\cs[0cm]{usepackage\{argumentation\}}\\

You can then create a new \texttt{af} environment in which the argumentation framework can then be built:\\

\noindent\cs[0cm]{begin\{af\}}

\quad\opt[green]{environment~contents}

\noindent\cs[0cm]{end\{af\}}\\

You may want to wrap the \texttt{af} environment in a \texttt{figure} environment in order to add a caption and reference label.
You can add a label inside the \texttt{af} environment via \verb|\label{|\opt[green]{label}\verb|}|.
Anywhere in your document, you can then reference the af with \verb|\ref{|\opt[green]{label}\verb|}|.

Inside the \texttt{af} environment, you can then add an argument as follows:\\

\noindent\cs[0cm]{argument\{\opt{name}\}}\\

Here, \opt{name} is the name of the argument displayed in the graph and the argument is automatically assigned an \emph{identifier} of the form: $a1$, $a2$, \dots.

To properly add further arguments, you also need to specify a position.
The \argumentation package offers two easy ways of doing that:\\

\noindent\cs[0cm]{argument[\opt{dir}=of \opt{argId}]\{\opt{name}\}}\\

\noindent\cs[0cm]{argument\{\opt{name}\} at (\opt{posX},\opt{posY})}\\

The first instance is \emph{relative positioning} where \opt{dir} is the direction of placement relative to the argument with the identifier \opt{argId}, with \opt{dir} typically being one of: \textsf{right}, \textsf{left}, \textsf{above}, \textsf{below}.

The second instance is \emph{absolute positioning} where \texttt{(\opt{posX}, \opt{posY})} is a set of coordinates, for example something like $\texttt{(2, 0)}$, $\texttt{(0, -2)}$ or $\texttt{(-1, 3.5)}$.

The next step is adding attacks.
For that you can simply use the following command:\\

\noindent\cs[0cm]{attack\{\opt{a1}\}\{\opt{a2}\}}\\

\noindent
Substitute \opt{a1} and \opt{a2} with the identifier of the two arguments.
Alternatively, you can also directly create bidirectional attacks and self-attacks with the following two commands:\\

\noindent\cs[0cm]{dualattack\{\opt{a1}\}\{\opt{a2}\}}\\
\noindent\cs[0cm]{selfattack\{\opt{a1}\}}\\


To customize the look of the arguments and attacks and for a detailed overview over all options and commands provided by this package, please refer to the following example or to the full documentation in Section~\ref{sec:documentation}.


\newpage
%%%%%%%%%%%%%%%%%%%%%%%%%%%%%%%%%%%%%%%%%%%%%%%%%%%%%%%%%%%%%%%%%%%%%%%%%%%%%%%%%%%%%%%%%%%%%%%%%%%%%%%%%%%
\section{Example Usage}\label{sec:example}
%%%%%%%%%%%%%%%%%%%%%%%%%%%%%%%%%%%%%%%%%%%%%%%%%%%%%%%%%%%%%%%%%%%%%%%%%%%%%%%%%%%%%%%%%%%%%%%%%%%%%%%%%%%
\vspace{-0.5cm}
\begin{figure}[ht]
    \centering
    \begin{af}[namestyle=math]
        \argument{a}
        \argument[right=of a1]{b}
        \argument[right=of a2]{c}
        \argument[rejected,right=of a3]{d}
        \argument[right=of a4,incomplete]{e}
        \argument[below=of a1]{f}
        \argument[inactive,right=of a6]{g}
        \argument[accepted,right=of a7]{h}
        \argument[right=of a8]{i}
        \argument[right=of a9]{j}

        \annotation[right,yshift=-0.4cm]{a5}{$a\lor b$}

        \afname{$F_{\ref{af:ex2}}$} at (-1,-1)

        \selfattack{a1}
        \dualattack{a1}{a6}
        \dualattack[inactive]{a6}{a7}
        
        \attack[inactive]{a8}{a7}
        \attack[inactive]{a7}{a2}
        \annotatedattack[above]{a3}{a2}{$3$}
        \annotatedattack[below,incomplete]{a4}{a5}{$x$}
        \attack{a5}{a10}
        \attack{a10}{a9}
        \attack{a9}{a4}

        \support{a4}{a3}
        \support{a9}{a3}
        \label{af:ex2}
    \end{af}
    \caption{The AF $F_{\ref{af:ex2}}$ created with the \argumentation package.}
    \label{fig:example}
\end{figure}

\begin{verbatim}
\usepackage[namestyle=math]{argumentation}
...
\begin{document}
...
\begin{figure}[ht]
    \centering
    \begin{af}
        \argument{a}
        \argument[right=of a1]{b}
        \argument[right=of a2]{c}
        \argument[rejected,right=of a3]{d}
        \argument[right=of a4,incomplete]{e}
        \argument[below=of a1]{f}
        \argument[inactive,right=of a6]{g}
        \argument[accepted,right=of a7]{h}
        \argument[right=of a8]{i}
        \argument[right=of a9]{j}

        \annotation[right,yshift=-0.4cm]{a5}{$a\lor b$}
        \afname{$F_{\ref{af:ex2}}$} at (-1,-1)

        \selfattack{a1}
        \dualattack{a1}{a6}
        \dualattack[inactive]{a6}{a7}
        
        \attack[inactive]{a8}{a7}
        \attack[inactive]{a7}{a2}
        \attack{a5}{a10}
        \attack{a10}{a9}
        \attack{a9}{a4}

        \annotatedattack[above]{a3}{a2}{$3$}
        \annotatedattack[below,incomplete]{a4}{a5}{$x$}
        \support{a4}{a3}
        \support{a9}{a3}
        \label{af:ex2}
    \end{af}
    \caption{The AF $F_{\ref{af:ex2}}$ created with the \argumentation package.}
    \label{fig:example}
\end{figure}
...
\end{document}
\end{verbatim}

\newpage
%%%%%%%%%%%%%%%%%%%%%%%%%%%%%%%%%%%%%%%%%%%%%%%%%%%%%%%%%%%%%%%%%%%%%%%%%%%%%%%%%%%%%%%%%%%%%%%%%%%%%%%%%%%
\section{Documentation for Version 1.4 [2024/10/31]}\label{sec:documentation}
%%%%%%%%%%%%%%%%%%%%%%%%%%%%%%%%%%%%%%%%%%%%%%%%%%%%%%%%%%%%%%%%%%%%%%%%%%%%%%%%%%%%%%%%%%%%%%%%%%%%%%%%%%%
The \argumentation package provides an easy way for creating argumentation frameworks\footnote{Dung, P. M. (1995). On the acceptability of arguments and its fundamental role in non-monotonic reasoning, logic programming and n-person games. Artificial intelligence.} in \LaTeX-documents.
It builds on the \tikzname package for drawing the argumentation graphs.
The \argumentation package provides simplified syntax while keeping the same customisation options and keeping full compatibility with all \tikzname features.
In addition to that, the \argumentation package provides the ability to label and reference the created argumentation frameworks as well as some other additional features.

The \argumentation package can be imported via the command\\

\noindent\cs{usepackage[\opt{options}]\{argumentation\}}\\

\vspace{-0.2cm}
In the following, we give an overview over the functionality of the \argumentation package.
Most importantly, that includes the \texttt{af} environment to encapsulate the created argumentation frameworks, the command \verb|\argument{ }| to create argument nodes and the \verb|attack{ }{ }| command to create attack edges.
Options to customise the appearance of arguments and attacks are described in Section~\ref{sec:options}.

\vspace{-0.2cm}
%%%%%%%%%%%%%%%%%%%%%%%%%%%%%%%%%%%%%%%%%%%%%%%%%%%%%%%%%%%%%%%%%%%%%%%%%%%%%%%%%%%%%%%%%%%
\subsection{The \texttt{af} Environment}
%%%%%%%%%%%%%%%%%%%%%%%%%%%%%%%%%%%%%%%%%%%%%%%%%%%%%%%%%%%%%%%%%%%%%%%%%%%%%%%%%%%%%%%%%%%
The \argumentation package provides an environment for creating argumentation frameworks in \LaTeX-documents.\\

\cs{begin\{af\} [\opt{options}]}

    \hspace{-1cm}\opt[green]{environment~contents}
    
\cs[-1.6cm]{end\{af\}}\\

\vspace{-0.2cm}
The \texttt{af} environment supports referencing.
For that add the command \cs[0cm]{label\{\opt[green]{label}\}} anywhere inside an \texttt{af} environment.
The AFs are automatically numbered in ascending order of occurrence.
The \opt[green]{label} allows you to reference the corresponding AF via \cs[0cm]{ref\{\opt[green]{label}\}} anywhere in the document.

If you want to create an AF that is excluded from the automatic numbering, the \argumentation package provides the \texttt{af*} version of the environment, which has the same functionality otherwise:\\

\cs{begin\{af*\} [\opt{options}]}

    \hspace{-1cm}\opt[green]{environment~contents}
    
\cs[-1.6cm]{end\{af*\}}\\

\vspace{-0.2cm}
The \texttt{af} (and \texttt{af*}) environment also accepts the package style options (see Section~\ref{sec:options}).
Locally set style options override defaults and the values set globally with the package import.

In general, the \texttt{af} environment extends the \texttt{tikzpicture} environment, meaning all \tikzname commands and parameters can be used for the \texttt{af} environment.
The \argumentation package also provides the options \textsf{small} or \textsf{tiny} for the \texttt{af} environment to create smaller AFs.
This is especially useful for two-column layout documents.


\begin{figure}[!ht]
    \begin{subfigure}{0.48\textwidth}
        \centering
        \begin{af}[small,namestyle=math]
            \label{af:small}
            \argument{a}
            \argument[right=of a1]{b}
            \argument[right=of a2]{c}
            \argument[right=of a3]{d}
            \argument[right=of a4]{e}
            \argument[below=of a1]{f}
            \argument[inactive,right=of a6]{g}
            \argument[inactive,argin,right=of a7]{h}
            \argument[right=of a8]{i}
            \argument[right=of a9]{j}
    
            \afname[left of=a1,yshift=-0.55cm,xshift=-0.2cm]{\afref{af:small}}
    
            \selfattack{a1}
            \dualattack[]{a1}{a6}
            \dualattack[inactive]{a6}{a7}
            \attack[inactive]{a8}{a7}
    
            \attack[inactive]{a7}{a2}
            \attack[]{a3}{a2}
    
            \support[]{a4}{a3}
            \support[]{a9}{a3}
    
            \attack[]{a4}{a5}
            \attack[]{a5}{a10}
            \attack[]{a10}{a9}
            \attack[]{a9}{a4}
        \end{af}
        \caption{An AF created with the \textsf{small} option set.}
        \label{fig:example_small}
    \end{subfigure}
    \hfill
    \begin{subfigure}{0.48\textwidth}
        \centering
        \begin{af}[tiny,namestyle=math]
            \label{af:tiny}
            \argument{a}
            \argument[right=of a1]{b}
            \argument[right=of a2]{c}
            \argument[right=of a3]{d}
            \argument[right=of a4]{e}
            \argument[below=of a1]{f}
            \argument[inactive,right=of a6]{g}
            \argument[inactive,argin,right=of a7]{h}
            \argument[right=of a8]{i}
            \argument[right=of a9]{j}
    
            \afname[left of=a1,yshift=-0.4cm,xshift=-0.2cm]{\afref{af:tiny}}
    
            \selfattack{a1}
            \dualattack[]{a1}{a6}
            \dualattack[inactive]{a6}{a7}
            \attack[inactive]{a8}{a7}
    
            \attack[inactive]{a7}{a2}
            \attack[]{a3}{a2}
    
            \support[]{a4}{a3}
            \support[]{a9}{a3}
    
            \attack[]{a4}{a5}
            \attack[]{a5}{a10}
            \attack[]{a10}{a9}
            \attack[]{a9}{a4}
        \end{af}
        \caption{An AF created with the \textsf{tiny} option set.}
        \label{fig:example_tiny}
    \end{subfigure}
    \caption{Argumentation frameworks using the \textsf{small} and \textsf{tiny} option of the \texttt{af} environment.}
    \label{fig:mini_afs}
\end{figure}


\newpage
%%%%%%%%%%%%%%%%%%%%%%%%%%%%%%%%%%%%%%%%%%%%%%%%%%%%%%%%%%%%%%%%%%%%%%%%%%%%%%%%%%%%%%%%%%%
\subsection{Creating Arguments}\label{sec:arguments}
%%%%%%%%%%%%%%%%%%%%%%%%%%%%%%%%%%%%%%%%%%%%%%%%%%%%%%%%%%%%%%%%%%%%%%%%%%%%%%%%%%%%%%%%%%%

\cs{argument [\opt{options}] (\opt{id}) \{\opt{name}\} at (\opt{posX},\opt{posY})}

\begin{description}
    \item[\opt{options}] (optional) a list of \tikzname style parameters and/or relative positioning information.
    \item[\opt{id}] (optional) the identifier of the new argument. Per default, when omitted, arguments will automatically be assigned an identifier of the form: $a1, a2, a3,...$.
    \item[\opt{name}] the displayed name of the argument.
    \item[\opt{posX},\opt{posY}] (optional) the coordinates where the argument is placed. Must be omitted if relative positioning is used.
\end{description}

%%%%%%%%%%%%%%%%%%%%%%%%%%%%%%%%%%%%%%%%%%%%%%%%%%%%%%%%%%%%%%%%%%%%%%%%%%%%%%%%%%%%%%%%%%%
\subsubsection{Positioning}\label{sec:positioning}
%%%%%%%%%%%%%%%%%%%%%%%%%%%%%%%%%%%%%%%%%%%%%%%%%%%%%%%%%%%%%%%%%%%%%%%%%%%%%%%%%%%%%%%%%%%
\todo{rewrite}
Placement of argument nodes with the \argumentation package relies on relative placement via the \tikzname-library \textsf{positioning}.
    The relative positioning information is provided as an optional parameter as follows\\

\cs{argument[\opt{dir}=of \opt{argId}]\{\opt{name}\}}\\

Additionally, you can adjust the horizontal/vertical position of an argument via the options \verb|xshift=|\opt{v} and \verb|yshift=|\opt{v}.
        The value \opt{v} is the horizontal/vertical offset, e.\,g., \verb|-6.6ex| or \verb|1cm|.

\begin{example}
    \todo{Example for creating arguments with different positioning}
\end{example}

\todo{where to put this?}
\subsubsection{Additional Argument Styling}
Furthermore, you can provide optional parameters to adjust the style of the argument node.
For that you can use all \tikzname-style options and additionally the following predefined style parameters (Refer to Appendix~\ref{sec:style-definitions} for a detailed description of these parameters):
\begin{align*}
    \mathsf{inactive} &\quad \text{The argument is displayed with grey outline and text.}\\
    \mathsf{incomplete} &\quad \text{The argument is displayed with a dotted outline.}\\
    \mathsf{invisible} &\quad \text{The argument node is completely transparent.}\\
    \mathsf{accepted} &\quad \text{The argument is displayed with green background color.}\\
    \mathsf{rejected} &\quad \text{The argument is displayed with red background color.}\\
    \mathsf{undecided} &\quad \text{The argument is displayed with cyan background color.}\\
    \mathsf{highlight} &\quad \text{The argument is displayed with yellow background color.}\\
\end{align*}

%%%%%%%%%%%%%%%%%%%%%%%%%%%%%%%%%%%%%%%%%%%%%%%%%%%%%%%%%%%%%%%%%%%%%%%%%%%%%%%%%%%%%%%%%%%
\subsection{Creating Attacks}\label{sec:attacks}
%%%%%%%%%%%%%%%%%%%%%%%%%%%%%%%%%%%%%%%%%%%%%%%%%%%%%%%%%%%%%%%%%%%%%%%%%%%%%%%%%%%%%%%%%%%

\cs{attack [\opt{options}] \{\opt{argId1}\} \{\opt{argId2}\}}
\begin{description}
    \item[\opt{arg1}] Identifier of the attacking argument.
    \item[\opt{arg2}] Identifier of the attacked argument.
\end{description}

\cs{dualattack [\opt{options}] \{\opt{argId1}\} \{\opt{argId2}\}}
\begin{description}
    \item[\opt{arg1}] Identifier of the first argument.
    \item[\opt{arg2}] Identifier of the second argument.
\end{description}


\cs{selfattack [\opt{options}] \{\opt{argId}\}}
\begin{description}
    \item[\opt{arg1}] Identifier of the self-attacking argument.
\end{description}

\cs{annotatedattack [\opt{options}] \{\opt{argId1}\} \{\opt{argId2}\} \{\opt{value}\}}
\begin{description}
    \item[\opt{options}] \todo{fix}Specifies where the annotation should be placed relative to the attack arrow and should be one of: \textsf{above}, \textsf{below}, \textsf{left}, \textsf{right}.
    \item[\opt{arg1}] Identifier of the attacking argument.
    \item[\opt{arg2}] Identifier of the attacked argument.
    \item[\opt{value}] The text that is annotated.
\end{description}

\cs{support [\opt{options}] \{\opt{argId1}\} \{\opt{argId2}\}}
\begin{description}
    \item[\opt{arg1}] Identifier of the supporting argument.
    \item[\opt{arg2}] Identifier of the supported argument.
\end{description}

\todo{where to put this?}
To customize an attack you can provide additional optional parameters:
\begin{align*}
    \mathsf{inactive} &\quad \text{The attack is displayed in grey.}\\
    \mathsf{incomplete} &\quad \text{The attack is displayed with a dotted line.}\\
    \mathsf{invisible} &\quad \text{The attack is completely transparent.}\\
    \mathsf{selfattack} &\quad \text{Use if source and target of the attack are the same node.}\\
    \mathsf{bend~right} &\quad \text{The attack arrow is bent to the right.}\\
    &\quad \text{Can additionally provide the angle, e.\,g., \textsf{bend~right=40}.}\\
    \mathsf{bend~left} &\quad \text{The attack arrow is bent to the left. Can also provide an angle.}
\end{align*}

\begin{example}
    \todo{Example for creating attacks (and other edges)}
\end{example}

%%%%%%%%%%%%%%%%%%%%%%%%%%%%%%%%%%%%%%%%%%%%%%%%%%%%%%%%%%%%%%%%%%%%%%%%%%%%%%%%%%%%%%%%%%%
\subsection{Beamer}\label{sec:beamer}
%%%%%%%%%%%%%%%%%%%%%%%%%%%%%%%%%%%%%%%%%%%%%%%%%%%%%%%%%%%%%%%%%%%%%%%%%%%%%%%%%%%%%%%%%%%

\cs{afreduct \{\opt{af-label}\} \{\opt{argument list}\}}

\cs{afrestriction \{\opt{af-label}\} \{\opt{argument list}\}}

\cs{afextension \{\opt{af-label}\} \{\opt{argument list}\}}

\cs{aflabeling \{\opt{af-label}\} \{\opt{argument list}\}}

\begin{example}
    \todo{Example of all four beamer functions}
\end{example}

%%%%%%%%%%%%%%%%%%%%%%%%%%%%%%%%%%%%%%%%%%%%%%%%%%%%%%%%%%%%%%%%%%%%%%%%%%%%%%%%%%%%%%%%%%%
\subsection{Other Commands}
%%%%%%%%%%%%%%%%%%%%%%%%%%%%%%%%%%%%%%%%%%%%%%%%%%%%%%%%%%%%%%%%%%%%%%%%%%%%%%%%%%%%%%%%%%%

\cs{annotation [\opt{options}] \{\opt{argId}\} \{\opt{value}\}}

\cs{afname [\opt{options}] (\opt{id}) \{\opt{name}\} at (\opt{posX}, \opt{posY})}

\cs{setargumentstyle \{\opt{style parameters}\}}

\cs{setatackstyle \{\opt{style parameters}\}}

\cs{setsupportstyle \{\opt{style parameters}\}}

\cs{setargumentcolorscheme \{\opt{style parameters}\}}

\cs{setafstyle \{\opt{style parameters}\}}

\begin{example}
    \todo{Example of the other commands}
\end{example}

%%%%%%%%%%%%%%%%%%%%%%%%%%%%%%%%%%%%%%%%%%%%%%%%%%%%%%%%%%%%%%%%%%%%%%%%%%%%%%%%%%%%%%%%%%%
\subsubsection{Argumentation Macros}\label{sec:macros}
%%%%%%%%%%%%%%%%%%%%%%%%%%%%%%%%%%%%%%%%%%%%%%%%%%%%%%%%%%%%%%%%%%%%%%%%%%%%%%%%%%%%%%%%%%%
\todo{rewrite}
Additionally, the following commands are provided to facilitate referencing argumentation frameworks.
To activate them, add the parameter \textsf{macros=true} when loading the package.
Most importantly the command \verb|\afref{|\opt{name}\verb|}| which works like the \verb|ref| command but adds the reference number directly into the index of the \verb|\AF| symbol.
You may redefine any of the first four commands if you prefer a different naming scheme for AFs.

\begin{table}[ht]
    \centering
    \begin{tabular}{lll}
        \verb|\AF|&& \AF \\
        \verb|\arguments|&& \arguments\\
        \verb|\attacks|&&\attacks\\
        \verb|\AFcomplete|&&\AFcomplete\\
        \verb|\afref{af:ex1}|&&\afref{af:ex1}\\
        \verb|\fullafref{af:ex1}|&\qquad\qquad\qquad&\fullafref{af:ex1}
    \end{tabular}
    \caption{Provided macros and their respective output.}
    \label{tab:macros}
\end{table}

\newpage
%%%%%%%%%%%%%%%%%%%%%%%%%%%%%%%%%%%%%%%%%%%%%%%%%%%%%%%%%%%%%%%%%%%%%%%%%%%%%%%%%%%%%%%%%%%%%%%%%%%%%%%%%%%
\section{Package Options}\label{sec:options}
%%%%%%%%%%%%%%%%%%%%%%%%%%%%%%%%%%%%%%%%%%%%%%%%%%%%%%%%%%%%%%%%%%%%%%%%%%%%%%%%%%%%%%%%%%%%%%%%%%%%%%%%%%%
The \argumentation package comes with some package options to customize the appearance of the created argumentation frameworks as well as some additional features.
All style package options can both be set globally when importing the package and also locally for each \texttt{af} environment.
To import the \argumentation package, use the following command in the preamble of your \LaTeX-document:\\

\noindent
\cs[0cm]{usepackage[\opt{options}]\{argumentation\}}\\

The following package options are currently available:

\begin{description}
    \item[argumentstyle] (default \texttt{standard}) Globally sets the appearance of the argument nodes.
    The \argumentation package provides five options: \texttt{standard}, \texttt{large}, \texttt{thick}, \texttt{gray} and \texttt{colored}.
    Detailed descriptions of these options can be found below.
    
    \item[attackstyle] (default \texttt{standard}) Globally sets the appearance of the attack edges.
    The package comes with three available options: \texttt{standard}, \texttt{large} and \texttt{modern}.
    Detailed descriptions of these options can be found below.
    
    \item[supportstyle] (default \texttt{standard}) Globally sets the appearance of the support edges.
    The package comes with three available options: \texttt{standard}, \texttt{dashed} and \texttt{double}.
    Detailed descriptions of these options can be found below.
    
    \item[namestyle] (default \texttt{none}) Sets the text formatting applied to the argument names in the document.
    The package comes with five available options: \texttt{none}, \texttt{math}, \texttt{bold}, \texttt{monospace} and \texttt{monoemph}.
    Detailed descriptions of these options can be found below.
    
    \item[indexing]  (default \texttt{numeric}) Enables or disables automatic generation of \tikzname node-IDs for the created arguments.
    The available options are: \texttt{none}, \texttt{numeric} and \texttt{alphabetic}.
    Under the default numeric indexing the generated argument IDs are of the form $a1, a2, \dots$.
    With alphabetic indexing the IDs will simply be letters: $a,b,\dots$.
    If \texttt{none} is selected, no IDs will be generated and you are required to provide them for each argument via the parameter \texttt{(\opt{id})} of the \verb|\argument| command.
    
    \item[macros] Boolean (default \texttt{false}) When enabled provides additional macros for naming and referencing argumentation frameworks (see Table~\ref{tab:macros}).
    \item[beamer] Boolean (default \texttt{false}) When enabled, provides the commands for recreating argumentations frameworks described in Section~\ref{sec:beamer}.
\end{description}

In the following we give an overview of the different options for the style parameters that can be used to customise the created argumentation frameworks.
For the exact definitions of these parameters, refer to Section~\ref{sec:parameters}.

\newpage
\paragraph{\sffamily argumentstyle=\opt{option}}
\begin{align*}
    \mathsf{standard} &\quad \text{Circular argument node with normal size argument name.}\\
    \mathsf{large} &\quad \text{Larger font of the argument name.}\\
    \mathsf{thick} &\quad \text{Thick black outline and normal size argument name.}\\
    \mathsf{gray} &\quad \text{Thick gray outline, light gray background.}\\
    \mathsf{colored} &\quad \text{Thick blue outline, light blue background.}
\end{align*}

\begin{figure}[!h]
    \begin{subfigure}{0.32\textwidth}
        \centering
        \begin{af}[argumentstyle=standard]
            \argument{a}
            \argument[right=of a1]{b}
    
            \attack[]{a1}{a2}
            \label{af:test}
        \end{af}
        \caption{\textsf{argumentstyle}=\textit{standard}}
        \label{fig:argumentstyle_standard}
    \end{subfigure}
    \hfill
    \begin{subfigure}{0.3\textwidth}
        \centering
        \begin{af}[argumentstyle=large]
            \argument{a}
            \argument[right=of a1]{b}
    
            \attack[]{a1}{a2}
        \end{af}
        \caption{\textsf{argumentstyle}=\textit{large}}
        \label{fig:argumentstyle_large}
    \end{subfigure}
    \hfill
    \begin{subfigure}{0.3\textwidth}
        \centering
        \begin{af}[argumentstyle=thick]
            \argument{a}
            \argument[right=of a1]{b}
    
            \attack[]{a1}{a2}
        \end{af}
        \caption{\textsf{argumentstyle}=\textit{thick}}
        \label{fig:argumentstyle_thick}
    \end{subfigure}

    \par\bigskip

    \begin{subfigure}{0.49\textwidth}
        \centering
        \begin{af}[argumentstyle=gray]
            \argument{a}
            \argument[right=of a1]{b}
    
            \attack[]{a1}{a2}
        \end{af}
        \caption{\textsf{argumentstyle}=\textit{gray}}
        \label{fig:argumentstyle_gray}
    \end{subfigure}
    \hfill
    \begin{subfigure}{0.49\textwidth}
        \centering
        \begin{af}[argumentstyle=colored]
            \argument{a}
            \argument[right=of a1]{b}
    
            \attack[]{a1}{a2}
        \end{af}
        \caption{\textsf{argumentstyle}=\textit{colored}}
        \label{fig:argumentstyle_colored}
    \end{subfigure}
    
    \caption{Available options for \textsf{argumentstyle}.}
    \label{fig:argumentstyle}
\end{figure}


\paragraph{\sffamily attackstyle=\opt{option}}
    
\begin{align*}
    \mathsf{standard} &\quad \text{Standard 'stealth' \tikzname arrow tip.}\\
    \mathsf{large} &\quad \text{Arrow tip is larger and sharper.}\\
    \mathsf{modern} &\quad \text{\tikzname ModernCS arrow tip.}
\end{align*}

\begin{figure}[!h]
    \begin{subfigure}{0.32\textwidth}
        \centering
        \begin{af}[attackstyle=standard]
            \argument{a}
            \argument[right=of a1]{b}
    
            \attack[]{a1}{a2}
        \end{af}
        \caption{\textsf{attackstyle}=\textit{standard}}
        \label{fig:attackstyle_standard}
    \end{subfigure}
    \hfill
    \begin{subfigure}{0.3\textwidth}
        \centering
        \begin{af}[attackstyle=large]
            \argument{a}
            \argument[right=of a1]{b}
    
            \attack[]{a1}{a2}
        \end{af}
        \caption{\textsf{attackstyle}=\textit{large}}
        \label{fig:attackstyle_large}
    \end{subfigure}
    \hfill
    \begin{subfigure}{0.3\textwidth}
        \centering
        \begin{af}[attackstyle=modern]
            \argument{a}
            \argument[right=of a1]{b}
    
            \attack[]{a1}{a2}
        \end{af}
        \caption{\textsf{attackstyle}=\textit{modern}}
        \label{fig:attackstyle_thick}
    \end{subfigure}
    \caption{Available options for \textsf{attackstyle}.}
    \label{fig:attackstyle}
\end{figure}


\paragraph{\sffamily supportstyle=\opt{option}}
    
\begin{align*}
    \mathsf{standard} &\quad \text{Same tip as attack arrow, perpendicular mark on arrow line.}\\
    \mathsf{dashed} &\quad \text{Dashed arrow line and same tip as attack arrow.}\\
    \mathsf{double} &\quad \text{Double arrow line and large flat tip.}
\end{align*}

\begin{figure}[!h]
    \begin{subfigure}{0.32\textwidth}
        \centering
        \begin{af}[supportstyle=standard]
            \argument{a}
            \argument[right=of a1]{b}
    
            \support[]{a1}{a2}
        \end{af}
        \caption{\textsf{supportstyle}=\textit{standard}}
        \label{fig:supportstyle_standard}
    \end{subfigure}
    \hfill
    \begin{subfigure}{0.3\textwidth}
        \centering
        \begin{af}[supportstyle=dashed]
            \argument{a}
            \argument[right=of a1]{b}
    
            \support[]{a1}{a2}
        \end{af}
        \caption{\textsf{supportstyle}=\textit{dashed}}
        \label{fig:supportstyle_dashed}
    \end{subfigure}
    \hfill
    \begin{subfigure}{0.3\textwidth}
        \centering
        \begin{af}[supportstyle=double]
            \argument{a}
            \argument[right=of a1]{b}
    
            \support[]{a1}{a2}
        \end{af}
        \caption{\textsf{supportstyle}=\textit{double}}
        \label{fig:supportstyle_double}
    \end{subfigure}
    \caption{Available options for \textsf{supportstyle}. Note that for \textit{standard} and \textit{dashed} the arrow tip of the selected \textsf{attackstyle} will be used.}
    \label{fig:supportstyle}
\end{figure}


\paragraph{\sffamily namestyle=\opt{option}}
    
\begin{align*}
    \mathsf{none} &\quad \text{No effect applied to argument name.}\\
    \mathsf{math} &\quad \text{The argument name is rendered as $math$ text.}\\
    &\quad\quad \text{(name must be given without mathmode).}\\
    \mathsf{bold} &\quad \text{The argument name is rendered in $\boldsymbol{bold}$.}\\
    &\quad\quad \text{(name must be given without mathmode).}\\
    \mathsf{monospace} &\quad \text{The argument name is rendered in \texttt{monospace} font.}\\
    &\quad\quad \text{(name must be given without mathmode).}\\
    \mathsf{monoemph} &\quad \text{The argument name is rendered as {\ttfamily\emph{name}}.}
\end{align*}

\begin{figure}[!ht]
    \begin{subfigure}{0.32\textwidth}
        \centering
        \begin{af}[namestyle=none]
            \argument{a}
            \argument[right=of a1]{b}
    
            \attack[]{a1}{a2}
        \end{af}
        \caption{\textsf{namestyle}=\textit{none}}
        \label{fig:namestyle_none}
    \end{subfigure}
    \hfill
    \begin{subfigure}{0.32\textwidth}
        \centering
        \begin{af}[namestyle=math]
            \argument{a}
            \argument[right=of a1]{b}
    
            \attack[]{a1}{a2}
        \end{af}
        \caption{\textsf{namestyle}=\textit{math}}
        \label{fig:namestyle_math}
    \end{subfigure}
    \hfill
    \begin{subfigure}{0.32\textwidth}
        \centering
        \begin{af}[namestyle=bold]
            \argument{a}
            \argument[right=of a1]{b}
    
            \attack[]{a1}{a2}
        \end{af}
        \caption{\textsf{namestyle}=\textit{bold}}
        \label{fig:namestyle_bold}
    \end{subfigure}
    \hfill
    \par\bigskip
    \begin{subfigure}{0.49\textwidth}
        \centering
        \begin{af}[namestyle=monospace]
            \argument{a}
            \argument[right=of a1]{b}
    
            \attack[]{a1}{a2}
        \end{af}
        \caption{\textsf{namestyle}=\textit{monospace}}
        \label{fig:namestyle_monospace}
    \end{subfigure}
    \hfill
    \begin{subfigure}{0.49\textwidth}
        \centering
        \begin{af}[namestyle=monoemph]
            \argument{a}
            \argument[right=of a1]{b}
    
            \attack[]{a1}{a2}
        \end{af}
        \caption{\textsf{namestyle}=\textit{monoemph}}
        \label{fig:namestyle_monoemph}
    \end{subfigure}
    \caption{Available options for \textsf{namestyle}. You can of course apply any formatting yourself when using the default \textsf{namestyle}=\textit{none}.}
    \label{fig:namestyle}
\end{figure}

\newpage
%%%%%%%%%%%%%%%%%%%%%%%%%%%%%%%%%%%%%%%%%%%%%%%%%%%%%%%%%%%%%%%%%%%%%%%%%%%%%%%%%%%%%%%%%%%%%%%%%%%%%%%%%%%
\section{Style Parameter Reference}\label{sec:parameters}
%%%%%%%%%%%%%%%%%%%%%%%%%%%%%%%%%%%%%%%%%%%%%%%%%%%%%%%%%%%%%%%%%%%%%%%%%%%%%%%%%%%%%%%%%%%%%%%%%%%%%%%%%%%
For reference, the style parameters provided by this package are listed below.
You may use or redefine them at your own discretion.

\begin{table}[ht]
        \centering
        \ttfamily
        \begin{tabular}{l|l}
            \toprule
            \tikzname-keyword & style parameters \\
            \midrule
            \textsf{argument size} & \emph{contains the currently selected argument size}\\
            \textsf{argument} & \emph{contains the currently selected argument style and size}\\
            \textsf{argument standard} & circle,inner sep=0,outer sep=0,draw=black\\
            \textsf{argument large} & circle,inner sep=0,outer sep=0,draw=black,font=\verb|\|large\\
            \textsf{argument thick} & circle,inner sep=0,outer sep=0,draw=black,line width=0.1em\\
            \textsf{argument gray} & argument thick,fill=gray!30,draw=gray!65,text=black!80\\
            \textsf{argument colored} & argument thick,fill=aigblue!40,draw=aigblue!80,text=black!80\\
            \midrule
            \textsf{attack} & \emph{contains the currently selected attack style}\\
            \textsf{attack standard} & -\{stealth'\}\\
            \textsf{attack large} & -\{Stealth[scale=1.25]\}\\
            \textsf{attack modern} & -\{To[sharp,length=0.65ex,line width=0.05em]\}\\
            \textsf{selfattack} & loop,min distance=0.4em,in=0,out=60,looseness=4.5\\
            \midrule
            \textsf{support} & \emph{contains the currently selected support style}\\
            \textsf{support standard} & attack,postaction=\{decorate,decoration=\{\dots\}\}\\
            \textsf{support dashed} & attack,densely dashed\\
            \textsf{support double} & -\{Classical TikZ Rightarrow\},double\\
            \midrule
            \textsf{inactive} & fill=none,draw=gray!50,text=gray!60\\
            \textsf{incomplete} & densely dashed\\
            \textsf{accepted} & fill=green!40\\
            \textsf{rejected} & fill=red!40\\
            \textsf{undecided} & fill=cyan!40\\
            \textsf{highlight} & fill=aigyellow!60\\
            \textsf{invisible} & draw=none,text=black!0,fill=none\\
            \midrule
            \textsf{standard}       & node distance=6.6ex,argument size/.style={minimum size=4.5ex},\\
                                    & attack width/.style={line width=0.05em}\\
            \textsf{small}          & node distance=3.5ex,argument size/.style={minimum size=3.4ex},\\
                                    & attack width/.style={line width=0.045em}\\
            \textsf{tiny}           & node distance=2.3ex,argument size/.style={minimum size=2.6ex,}\\
                                    & attack width/.style={line width=0.03em},font=\verb|\small|\\
            \bottomrule
        \end{tabular}
        \caption{Reference list of \tikzname-style parameters provided by the \argumentation package.}
        \label{tab:styles}
    \end{table}
\newpage
%%%%%%%%%%%%%%%%%%%%%%%%%%%%%%%%%%%%%%%%%%%%%%%%%%%%%%%%%%%%%%%%%%%%%%%%%%%%%%%%%%%%%%%%%%%%%%%%%%%%%%%%%%%
\section{Version History}\label{sec:history}
%%%%%%%%%%%%%%%%%%%%%%%%%%%%%%%%%%%%%%%%%%%%%%%%%%%%%%%%%%%%%%%%%%%%%%%%%%%%%%%%%%%%%%%%%%%%%%%%%%%%%%%%%%%
\subsection*{[v1.4 2024/10/31]}
\begin{itemize}
    \item Added functions \verb|\aflabeling|, \verb|\afextension|, \verb|\afreduct| and \verb|\afrestriction| that recreate (parts of) previously created argumentation frameworks. Can be enabled via the package option \textsf{beamer=true}.
    \item Added internal storage of arguments and attacks of an argumentation framework to enable further computations.
    \item Added environment \verb|af*| for argumentation frameworks that are unlabeled/uncounted.
    \item Added command \verb|\setargumentcolorscheme{ }{ }| to change color scheme of the \textsf{colored} argument style.
    \item Added command \verb|\setafstyle{ }| to set global style options for the AFs.
    \item Added optional parameter (\opt{value}) to \verb|\attack| command to add a label to the attack edge (undocumented for now).
    \item Major revision of the documentation.
    \item Various minor changes to internal functions, naming scheme and comments.
\end{itemize}
\subsection*{[v1.3 2024/09/25]}
\begin{itemize}
    \item Added support for \verb|\label{ }| and \verb|\ref{ }| to \texttt{af} environment.
    \item Added commands \verb|\AF|, \verb|\arguments|, \verb|\attacks| and \verb|\AFcomplete| to facilitate consistent naming of AFs. Have to be loaded with the package option \textsf{macros=true}.
    \item Added commands \verb|\afref{ }| and \verb|\fullafref{ }| to reference AFs.
    \item adjusted scaling of nodes and arrows for larger page sizes.
    \item added new style options for arguments.
    \item Various minor fixes and changes regarding the \textsf{namestyle} package option.
\end{itemize}

\subsection*{[v1.2 2024/06/07]}
\begin{itemize}
    \item Changed Syntax of \verb|\argument| command. The \textit{id} parameter is now given inside parenthesis instead of curly braces and is optional.
    \item Added absolute positioning to \verb|\argument| command, like for \tikzname nodes.
    \item Added package option $\textsf{indexing}$ to toggle automatic generation of identifiers for created argument nodes. Can be set to \textit{none}, or selected between \textit{alphabetic} and \textit{numeric} (default).
    \item All package style options can now also be set locally in the \texttt{af} environment.
    \item Adjusted \verb|\annotatedattack| to require position parameter.
    \item Various minor bugfixes regarding the \textsf{namestyle} package option.
    \item Added new argumentstyle \textsf{large}.
\end{itemize}

\subsection*{[v1.1 2023/12/03]}
\begin{itemize}
    \item Adjusted standard styles.
    \item Added command for creating annotated attacks.
    \item Now only provides one environment, which can be parameterised.
    \item Changed option management to pgfkeys.
    \item Updated and improved documentation.
\end{itemize}

\subsection*{[v1.0 2023/11/05]}
\begin{itemize}
    \item First Version.
\end{itemize}

\end{document}