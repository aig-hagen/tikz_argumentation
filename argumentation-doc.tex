\documentclass{scrartcl}

\usepackage[top=1.35in, bottom=1.33in, left=1.35in, right=1.35in]{geometry}
\usepackage[T1]{fontenc}                        % Font Encoding
\usepackage[utf8]{inputenc}                     % Input font encoding
\usepackage[english]{babel}                     % Language
\usepackage{subcaption}                         % Multi-part Figures
\usepackage[nohyperref]{doc}                    % Showing Commands (must load before tcolorbox)
\usepackage[breakable]{tcolorbox}               % Fancier Examples
\usepackage{booktabs}                           % Fancier Tables
\usepackage{enumitem}                           % Fancier Enumeration
\usepackage[hidelinks,                          % Setting Meta-Data and hyperlinks
            pdfauthor={Lars Bengel},
            pdftitle={The argumentation LaTeX-package},
            pdfsubject={Documentation of the argumentation LaTeX-package},
            pdfkeywords={argumentation,tikz,graphs}]{hyperref}
\usepackage[macros,beamer]{argumentation}       % Main Package
            
%%% Document settings
\KOMAoptions{
    paper=a4,
    fontsize=10pt,
    cleardoublepage=empty,
    footinclude=true
}

%%% Various commands used in the document
\newcommand{\todo}[1]{\textcolor{magenta}{TODO: #1}} % for todos
\newcommand{\tikzname}{Ti\emph{k}Z\xspace}
\newcommand{\argumentation}{\texttt{argumentation}\xspace}
\newtheorem{example}{Example}
\tcolorboxenvironment{example}{colframe=aigblue,colback=aigyellow!30,breakable,before skip=10pt,after skip=10pt}
\newcommand{\mopt}[2][red]{\textcolor{#1}{\ttfamily \meta{#2}}}
\newcommand{\oopt}[2][orange]{\textcolor{#1}{\ttfamily \meta{#2}}}
\newcommand{\popt}[2][magenta]{\textcolor{#1}{\ttfamily \meta{#2}}}
\newcommand{\sopt}[2][green]{\textcolor{#1}{\ttfamily \meta{#2}}}

%%% Taken from ltxdoc-class
\def\cmd#1{\cs{\expandafter\cmd@to@cs\string#1}}
\def\cmd@to@cs#1#2{\char\number`#2\relax}
\DeclareRobustCommand\cs[2][-1.6cm]{\hspace{#1}\texttt{\char`\\#2}}

\title{The \argumentation Package}
\author{Lars Bengel\footnotemark\\\small lars.bengel@fernuni-hagen.de}
\date{Version 1.6 [2025/07/31]}

\begin{document}
\maketitle

\renewcommand*{\thefootnote}{\fnsymbol{footnote}}
\footnotetext[1]{Please report any issues at \url{https://github.com/aig-hagen/tikz_argumentation}}
\renewcommand*{\thefootnote}{\arabic{footnote}}

\begin{minipage}{.26\textwidth}
    \centering
        \begin{af}[argumentstyle=gray,namestyle=monospace]
            \argument{a}
            \argument[right=of a1]{b}
            \argument[below=of a1]{c}
            \argument[right=of a3]{d}
    
            \attack{a1}{a2}
            \attack{a2}{a3}
            \attack[bend right]{a3}{a4}
            \label{af:example}
        \end{af}
\end{minipage}
\begin{minipage}{.5\textwidth}
    \begin{small}
    \begin{verbatim}
        \begin{af}[argumentstyle=gray,namestyle=monospace]
            \argument{a}
            \argument[right=of a1]{b}
            \argument[below=of a1]{c}
            \argument[right=of a3]{d}
    
            \attack{a1}{a2}
            \attack{a2}{a3}
            \attack[bend right]{a3}{a4}
            \label{af:example}
        \end{af}
    \end{verbatim}
    \end{small}
\end{minipage}

\tableofcontents

\newpage

%\sffamily
%%%%%%%%%%%%%%%%%%%%%%%%%%%%%%%%%%%%%%%%%%%%%%%%%%%%%%%%%%%%%%%%%%%%%%%%%%%%%%%%%%%%%%%%%%%%%%%%%%%%%%%%%%%
\section{Quick Guide}\label{sec:quick}
%%%%%%%%%%%%%%%%%%%%%%%%%%%%%%%%%%%%%%%%%%%%%%%%%%%%%%%%%%%%%%%%%%%%%%%%%%%%%%%%%%%%%%%%%%%%%%%%%%%%%%%%%%%
\noindent
To create an argumentation framework in your \LaTeX-document, you first have to import the \argumentation package in the preamble:\\

\noindent\cs[0cm]{usepackage\{argumentation\}}\\

You can then create a new \texttt{af} environment in which the argumentation framework can then be built:\\

\noindent\cs[0cm]{begin\{af\}}

\quad\sopt{environment~contents}

\noindent\cs[0cm]{end\{af\}}\\

You may want to wrap the \texttt{af} environment in a \texttt{figure} environment in order to add a caption and reference label.
You can also add a label inside the \texttt{af} environment via \verb|\label{|\sopt{label}\verb|}|.
Anywhere in your document, you can then reference the af with \verb|\ref{|\sopt{label}\verb|}|.

Inside the \texttt{af} environment, you can then add an argument as follows:\\

\noindent\cs[0cm]{argument\{\mopt{name}\}}\\

Here, \mopt{name} is the name of the argument displayed in the graph and the argument is automatically assigned an \emph{identifier} of the form: $a1$, $a2$, \dots.

To properly add further arguments, you also need to specify a position.
The \argumentation package offers two easy ways of doing that:\\

\noindent\cs[0cm]{argument[\mopt{dir}=of \mopt{argId}]\{\mopt{name}\}}\\

\noindent\cs[0cm]{argument\{\mopt{name}\} at (\popt{posX},\popt{posY})}\\

The first instance is \emph{relative positioning} where \mopt{dir} is the direction of placement relative to the argument with the identifier \mopt{argId}, with \mopt{dir} typically being one of: \textsf{right}, \textsf{left}, \textsf{above}, \textsf{below}.

The second instance is \emph{absolute positioning} where \texttt{(\popt{posX}, \popt{posY})} is a set of coordinates, for example something like $\texttt{(2, 0)}$, $\texttt{(0, -2)}$ or $\texttt{(-1, 3.5)}$.

The next step is adding attacks.
For that you can simply use the following command:\\

\noindent\cs[0cm]{attack\{\mopt{a1}\}\{\mopt{a2}\}}\\

\noindent
Substitute \mopt{a1} and \mopt{a2} with the identifier of the two arguments.
Alternatively, you can also directly create bidirectional attacks and self-attacks with the following two commands:\\

\noindent\cs[0cm]{dualattack\{\mopt{a1}\}\{\mopt{a2}\}}\\
\noindent\cs[0cm]{selfattack\{\mopt{a1}\}}\\


To customize the look of the arguments and attacks and for a detailed overview over all options and commands provided by this package, please refer to the following example or to the full documentation in Section~\ref{sec:documentation}.


\newpage
%%%%%%%%%%%%%%%%%%%%%%%%%%%%%%%%%%%%%%%%%%%%%%%%%%%%%%%%%%%%%%%%%%%%%%%%%%%%%%%%%%%%%%%%%%%%%%%%%%%%%%%%%%%
\section{Example Usage}\label{sec:example}
%%%%%%%%%%%%%%%%%%%%%%%%%%%%%%%%%%%%%%%%%%%%%%%%%%%%%%%%%%%%%%%%%%%%%%%%%%%%%%%%%%%%%%%%%%%%%%%%%%%%%%%%%%%
\vspace{-0.5cm}
\begin{figure}[ht]
    \centering
    \begin{af}[namestyle=math]
        \argument{a}
        \argument[right=of a1]{b}
        \argument[right=of a2]{c}
        \argument[rejected,right=of a3]{d}
        \argument[right=of a4,incomplete]{e}
        \argument[below=of a1]{f}
        \argument[inactive,right=of a6]{g}
        \argument[accepted,right=of a7]{h}
        \argument[right=of a8]{i}
        \argument[right=of a9]{j}

        \annotation[right,yshift=-0.4cm]{a5}{$a\lor b$}

        \afname{$F_{\ref{af:ex2}}$} at (-1,-1)

        \selfattack{a1}
        \dualattack{a1}{a6}
        \dualattack[inactive]{a6}{a7}
        
        \attack[inactive]{a8}{a7}
        \attack[inactive]{a7}{a2}
        \annotatedattack[above]{a3}{a2}{$3$}
        \annotatedattack[below,incomplete]{a4}{a5}{$x$}
        \attack{a5}{a10}
        \attack{a10}{a9}
        \attack{a9}{a4}

        \support{a4}{a3}
        \support{a9}{a3}
        \label{af:ex2}
    \end{af}
    \caption{The AF $F_{\ref{af:ex2}}$ created with the \argumentation package.}
    \label{fig:example}
\end{figure}

\begin{verbatim}
\usepackage[namestyle=math]{argumentation}
...
\begin{document}
...
\begin{figure}[ht]
    \centering
    \begin{af}
        \argument{a}
        \argument[right=of a1]{b}
        \argument[right=of a2]{c}
        \argument[rejected,right=of a3]{d}
        \argument[right=of a4,incomplete]{e}
        \argument[below=of a1]{f}
        \argument[inactive,right=of a6]{g}
        \argument[accepted,right=of a7]{h}
        \argument[right=of a8]{i}
        \argument[right=of a9]{j}

        \annotation[right,yshift=-0.4cm]{a5}{$a\lor b$}
        \afname{$F_{\ref{af:ex2}}$} at (-1,-1)

        \selfattack{a1}
        \dualattack{a1}{a6}
        \dualattack[inactive]{a6}{a7}
        
        \attack[inactive]{a8}{a7}
        \attack[inactive]{a7}{a2}
        \attack{a5}{a10}
        \attack{a10}{a9}
        \attack{a9}{a4}

        \annotatedattack[above]{a3}{a2}{$3$}
        \annotatedattack[below,incomplete]{a4}{a5}{$x$}
        \support{a4}{a3}
        \support{a9}{a3}
        \label{af:ex2}
    \end{af}
    \caption{The AF $F_{\ref{af:ex2}}$ created with the \argumentation package.}
    \label{fig:example}
\end{figure}
...
\end{document}
\end{verbatim}

\newpage
%%%%%%%%%%%%%%%%%%%%%%%%%%%%%%%%%%%%%%%%%%%%%%%%%%%%%%%%%%%%%%%%%%%%%%%%%%%%%%%%%%%%%%%%%%%%%%%%%%%%%%%%%%%
\section{Documentation for Version 1.6 [2025/07/31]}\label{sec:documentation}
%%%%%%%%%%%%%%%%%%%%%%%%%%%%%%%%%%%%%%%%%%%%%%%%%%%%%%%%%%%%%%%%%%%%%%%%%%%%%%%%%%%%%%%%%%%%%%%%%%%%%%%%%%%
The \argumentation package provides an easy way for creating argumentation frameworks\footnote{Dung, P. M. (1995). On the acceptability of arguments and its fundamental role in non-monotonic reasoning, logic programming and n-person games. Artificial intelligence.} in \LaTeX-documents.
It builds on the \tikzname package for drawing the argumentation graphs.
The \argumentation package provides simplified syntax while keeping the same customisation options and keeping full compatibility with all \tikzname features.
In addition to that, the \argumentation package provides the ability to label and reference the created argumentation frameworks as well as some other additional features.

The \argumentation package can be imported via the command\\

\noindent\cs{usepackage[\oopt{options}]\{argumentation\}}\\

\vspace{-0.2cm}
In the following, we give an overview over the functionality of the \argumentation package.
Most importantly, that includes the \texttt{af} environment to encapsulate the created argumentation frameworks, the command \verb|\argument{ }| to create argument nodes and the \verb|attack{ }{ }| command to create attack edges.
Options to customise the appearance of arguments and attacks are described in Section~\ref{sec:options}.

\vspace{-0.2cm}
%%%%%%%%%%%%%%%%%%%%%%%%%%%%%%%%%%%%%%%%%%%%%%%%%%%%%%%%%%%%%%%%%%%%%%%%%%%%%%%%%%%%%%%%%%%
\subsection{The \texttt{af} Environment}
%%%%%%%%%%%%%%%%%%%%%%%%%%%%%%%%%%%%%%%%%%%%%%%%%%%%%%%%%%%%%%%%%%%%%%%%%%%%%%%%%%%%%%%%%%%
The \argumentation package provides an environment for creating argumentation frameworks in \LaTeX-documents.\\

\noindent\cs{begin\{af\} [\oopt{options}]}

    \hspace{-1cm}\sopt{environment~contents}
    
\noindent\cs[-1.6cm]{end\{af\}}\\

\vspace{-0.2cm}
The \texttt{af} environment supports referencing.
For that add the command \cs[0cm]{label\{\sopt{label}\}} anywhere inside an \texttt{af} environment.
The AFs are automatically numbered in ascending order of occurrence.
The \sopt{label} allows you to reference the corresponding AF via \cs[0cm]{ref\{\sopt{label}\}} anywhere in the document.

If you want to create an AF that is excluded from the automatic numbering, the \argumentation package provides the \texttt{af*} version of the environment, which has the same functionality otherwise:\\

\noindent\cs{begin\{af*\} [\oopt{options}]}

    \hspace{-1cm}\sopt{environment~contents}
    
\noindent\cs[-1.6cm]{end\{af*\}}\\

\vspace{-0.2cm}
The \texttt{af} (and \texttt{af*}) environment also accepts the package style options (see Section~\ref{sec:options}).
Locally set style options override defaults and the values set globally with the package import.

In general, the \texttt{af} environment extends the \texttt{tikzpicture} environment, meaning all \tikzname commands and parameters can be used for the \texttt{af} environment.
The \argumentation package also provides the options \textsf{small} or \textsf{tiny} for the \texttt{af} environment to create smaller AFs.
This is especially useful for two-column layout documents.


\begin{figure}[!ht]
    \begin{subfigure}{0.48\textwidth}
        \centering
        \begin{af}[small,namestyle=math]
            \label{af:small}
            \argument{a}
            \argument[right=of a1]{b}
            \argument[right=of a2]{c}
            \argument[right=of a3]{d}
            \argument[right=of a4]{e}
            \argument[below=of a1]{f}
            \argument[inactive,right=of a6]{g}
            \argument[inactive,argin,right=of a7]{h}
            \argument[right=of a8]{i}
            \argument[right=of a9]{j}
    
            \afname[left of=a1,yshift=-0.55cm,xshift=-0.2cm]{\afref{af:small}}
    
            \selfattack{a1}
            \dualattack[]{a1}{a6}
            \dualattack[inactive]{a6}{a7}
            \attack[inactive]{a8}{a7}
    
            \attack[inactive]{a7}{a2}
            \attack[]{a3}{a2}
    
            \support[]{a4}{a3}
            \support[]{a9}{a3}
    
            \attack[]{a4}{a5}
            \attack[]{a5}{a10}
            \attack[]{a10}{a9}
            \attack[]{a9}{a4}
        \end{af}
        \caption{An AF created with the \textsf{small} option set.}
        \label{fig:example_small}
    \end{subfigure}
    \hfill
    \begin{subfigure}{0.48\textwidth}
        \centering
        \begin{af}[tiny,namestyle=math]
            \label{af:tiny}
            \argument{a}
            \argument[right=of a1]{b}
            \argument[right=of a2]{c}
            \argument[right=of a3]{d}
            \argument[right=of a4]{e}
            \argument[below=of a1]{f}
            \argument[inactive,right=of a6]{g}
            \argument[inactive,argin,right=of a7]{h}
            \argument[right=of a8]{i}
            \argument[right=of a9]{j}
    
            \afname[left of=a1,yshift=-0.4cm,xshift=-0.2cm]{\afref{af:tiny}}
    
            \selfattack{a1}
            \dualattack[]{a1}{a6}
            \dualattack[inactive]{a6}{a7}
            \attack[inactive]{a8}{a7}
    
            \attack[inactive]{a7}{a2}
            \attack[]{a3}{a2}
    
            \support[]{a4}{a3}
            \support[]{a9}{a3}
    
            \attack[]{a4}{a5}
            \attack[]{a5}{a10}
            \attack[]{a10}{a9}
            \attack[]{a9}{a4}
        \end{af}
        \caption{An AF created with the \textsf{tiny} option set.}
        \label{fig:example_tiny}
    \end{subfigure}
    \caption{Argumentation frameworks using the \textsf{small} and \textsf{tiny} option of the \texttt{af} environment.}
    \label{fig:mini_afs}
\end{figure}


\newpage
%%%%%%%%%%%%%%%%%%%%%%%%%%%%%%%%%%%%%%%%%%%%%%%%%%%%%%%%%%%%%%%%%%%%%%%%%%%%%%%%%%%%%%%%%%%
\subsection{Creating Arguments}\label{sec:arguments}
%%%%%%%%%%%%%%%%%%%%%%%%%%%%%%%%%%%%%%%%%%%%%%%%%%%%%%%%%%%%%%%%%%%%%%%%%%%%%%%%%%%%%%%%%%%

Inside an \texttt{af} (or \texttt{af*}) environment, you can create argument nodes for the argumentation framework with the following command\\

\noindent\cs{argument [\oopt{options}] (\popt{id}) \{\mopt{name}\} at (\popt{posX},\popt{posY})}

\begin{description}
    \item[\oopt{options}] (optional) a list of \tikzname style parameters and/or relative positioning information.
    \item[\popt{id}] (optional) the identifier of the new argument. Per default, when omitted, arguments will automatically be assigned an identifier of the form: $a1, a2, a3,...$.
    \item[\mopt{name}] the displayed name of the argument.
    \item[\popt{posX},\popt{posY}] (optional) the coordinates where the argument is placed. Must be omitted if relative positioning is used.
\end{description}

%%%%%%%%%%%%%%%%%%%%%%%%%%%%%%%%%%%%%%%%%%%%%%%%%%%%%%%%%%%%%%%%%%%%%%%%%%%%%%%%%%%%%%%%%%%
\subsubsection{Positioning}\label{sec:positioning}
%%%%%%%%%%%%%%%%%%%%%%%%%%%%%%%%%%%%%%%%%%%%%%%%%%%%%%%%%%%%%%%%%%%%%%%%%%%%%%%%%%%%%%%%%%%
The \argumentation package also provides the ability to use \emph{relative positioning} instead of absolute positioning via coordinates.
For that, it relies on relative placement via the \tikzname-library \textsf{positioning}.
The relative positioning information is provided as an optional parameter via \texttt{[\oopt{options}]} as follows\\

\noindent\cs{argument[\mopt{dir}=of \mopt{argId}]\{\mopt{name}\}}

\begin{description}
    \item[\mopt{dir}] The direction of placement relative to the argument \mopt{argId}. Typically one of: \textsf{above}, \textsf{right}, \textsf{left} or \textsf{below}.
    \item[\mopt{argId}] The identifier of another argument.
    \item[\mopt{name}] The displayed name of the argument.
\end{description}

Additionally, you can adjust the horizontal/vertical position of an argument by adding \verb|xshift=|\mopt{v} or \verb|yshift=|\mopt{v} inside the \texttt{[...]}.
The value \mopt{v} is hereby the horizontal/vertical offset, e.\,g., \verb|-6.6ex| or \verb|1cm|.

In the following, we list some useful style options for the \verb|\argument| command, provided by the\argumentation package:

\vspace{-0.15cm}
\begin{table}[ht]
    \centering
    \begin{tabular}{ll}
        \textsf{inactive} &\quad \text{The argument is displayed with grey outline and text.}\\
        \textsf{incomplete} &\quad \text{The argument is displayed with a dotted outline.}\\
        \textsf{invisible} &\quad \text{The argument node is completely transparent.}\\
        \textsf{accepted} &\quad \text{The argument is displayed with green background color.}\\
        \textsf{rejected} &\quad \text{The argument is displayed with red background color.}\\
        \textsf{undecided} &\quad \text{The argument is displayed with cyan background color.}\\
        \textsf{highlight} &\quad \text{The argument is displayed with yellow background color.}\\
    \end{tabular}
    \caption{Some style options for the \cs[0cm]{argument} command (for exact definition see Section~\ref{sec:parameters}).}
    \label{tab:argument_style}
\end{table}

\vspace{-0.4cm}
\begin{example}~

    \vspace{-0.2cm}
    \begin{minipage}{0.3\textwidth}
        \centering
        \begin{af}[namestyle=math]
            \argument{a}
            \argument{b} at (1,0)
            \argument[below=of a1,accepted](c){c}
            \argument(x){d} at (1,-1)
            \argument[right=of x,yshift=0.5cm]{e}
            \argument[inactive]{f} at (2,-1.5)
        \end{af}
    \end{minipage}%
    \begin{minipage}{0.7\textwidth}
        \begin{small}
        \begin{verbatim}
            \begin{af}[namestyle=math]
                \argument{a}
                \argument{b} at (1,0)
                \argument[below=of a1,accepted](c){c}
                \argument(x){d} at (1,-1)
                \argument[right=of x,yshift=0.5cm]{e}
                \argument[inactive]{f} at (2,-1.5)
            \end{af}
        \end{verbatim}
        \end{small}
    \end{minipage}
    \vspace{-0.3cm}
\end{example}


\newpage
%%%%%%%%%%%%%%%%%%%%%%%%%%%%%%%%%%%%%%%%%%%%%%%%%%%%%%%%%%%%%%%%%%%%%%%%%%%%%%%%%%%%%%%%%%%
\subsection{Creating Attacks}\label{sec:attacks}
%%%%%%%%%%%%%%%%%%%%%%%%%%%%%%%%%%%%%%%%%%%%%%%%%%%%%%%%%%%%%%%%%%%%%%%%%%%%%%%%%%%%%%%%%%%

To create an directed attack between two argument nodes, you can use the following command.
The parameters \mopt{arg1} and \mopt{arg2} are the identifiers of the two arguments.\\

\noindent\cs{attack [\oopt{options}] \{\mopt{argId1}\} \{\mopt{argId2}\}}
\begin{description}
    \item[\oopt{options}] (optional) a list of \tikzname style parameters.
    \item[\mopt{arg1}] Identifier of the attacking argument.
    \item[\mopt{arg2}] Identifier of the attacked argument.\\
\end{description}

To simplify creating special types of attacks, like bidirectional attacks or self-attacks, the following two commands are provided.\\

\noindent\cs{dualattack [\oopt{options}] \{\mopt{argId1}\} \{\mopt{argId2}\}}
\begin{description}
    \item[\oopt{options}] (optional) a list of \tikzname style parameters.
    \item[\mopt{arg1}] Identifier of the first argument.
    \item[\mopt{arg2}] Identifier of the second argument.
\end{description}

\noindent\cs{selfattack [\oopt{options}] \{\mopt{argId}\}}
\begin{description}
    \item[\oopt{options}] (optional) a list of \tikzname style parameters.
    \item[\mopt{arg1}] Identifier of the self-attacking argument.
\end{description}

For \verb|\selfattack| you might want to specify the position of the attack loop.
For that, you should provide the start and end point of the attack-edge (as a degree from \verb|0| to \verb|360|) via the optional \tikzname-parameters \texttt{in=\oopt{degree1}} and \texttt{out=\oopt{degree2}}.
The default values are \texttt{in=0} and \texttt{out=60}.\\

If you want to attach a value to an attack edge, you may use the following command.\\

\noindent\cs{annotatedattack [\oopt{options}] \{\mopt{argId1}\} \{\mopt{argId2}\} \{\mopt{value}\}}
\begin{description}
    \item[\oopt{options}] Optional \tikzname parameters. Must also include one of the following parameters to specify placement of the annotation relative to the attack arrow: \textsf{above}, \textsf{below}, \textsf{left}, \textsf{right}.
    \item[\mopt{arg1}] Identifier of the attacking argument.
    \item[\mopt{arg2}] Identifier of the attacked argument.
    \item[\mopt{value}] The text that is annotated.\\
\end{description}

As an alternative to the standard attack arrow, you can use the following command.\\

\noindent\cs{support [\oopt{options}] \{\mopt{argId1}\} \{\mopt{argId2}\}}
\begin{description}
    \item[\oopt{options}] (optional) a list of \tikzname style parameters.
    \item[\mopt{arg1}] Identifier of the supporting argument.
    \item[\mopt{arg2}] Identifier of the supported argument.\\
\end{description}

\newpage
Some useful style options for the attacks (and other edges) are listed below:

\begin{table}[ht]
    \centering
    \begin{tabular}{ll}
        \textsf{inactive} &\quad \text{The attack is displayed in grey.}\\
        \textsf{incomplete} &\quad \text{The attack is displayed with a dotted line.}\\
        \textsf{invisible} &\quad \text{The attack is completely transparent.}\\
        \textsf{selfattack} &\quad \text{Use if source and target of the attack are the same node.}\\
        \textsf{bend~right} &\quad \text{The attack arrow is bent to the right.}\\
        &\quad \text{Can additionally provide the angle, e.\,g., \textsf{bend~right=40}.}\\
        \textsf{bend~left} &\quad \text{The attack arrow is bent to the left. Can also provide an angle.}
    \end{tabular}
    \caption{Some useful style options for the \cs[0cm]{attack} (and related) commands. For the exact definition see Section~\ref{sec:parameters}.}
    \label{tab:attack_style}
\end{table}

\begin{example}~

    \begin{minipage}{0.3\textwidth}
        \centering
        \begin{af}[namestyle=math]
            \argument{a}
            \argument[right=of a1]{b}
            \argument[below=of a1]{c}
            \argument[right=of a3]{d}

            \attack{a3}{a1}
            \selfattack[incomplete]{a3}
            \dualattack{a1}{a2}
            \annotatedattack[right]{a2}{a4}{$x$}
            \support[bend left]{a4}{a3}
        \end{af}
    \end{minipage}%
    \begin{minipage}{0.7\textwidth}
        \begin{verbatim}
            \begin{af}[namestyle=math]
                ...
    
                \attack{a3}{a1}
                \selfattack[incomplete]{a3}
                \dualattack{a1}{a2}
                \annotatedattack[right]{a2}{a4}{$x$}
                \support[bend left]{a4}{a3}
            \end{af}
        \end{verbatim}
    \end{minipage}
\end{example}

%%%%%%%%%%%%%%%%%%%%%%%%%%%%%%%%%%%%%%%%%%%%%%%%%%%%%%%%%%%%%%%%%%%%%%%%%%%%%%%%%%%%%%%%%%%
\subsection{Beamer}\label{sec:beamer}
%%%%%%%%%%%%%%%%%%%%%%%%%%%%%%%%%%%%%%%%%%%%%%%%%%%%%%%%%%%%%%%%%%%%%%%%%%%%%%%%%%%%%%%%%%%
Most commands provided by the \argumentation package support \emph{overlay specifications} used for creating presentations with \textsf{beamer}\footnote{\url{https://tug.ctan.org/macros/latex/contrib/beamer/doc/beameruserguide.pdf}}.
To specify overlay information, use \verb|<X>| directly after the command name, where \verb|X| is for instance a number, a list of numbers or any other form of overlay specification (see \textsf{beamer} documentation for more information).
All commands for creating AFs introduced so far support overlay specifications.

\begin{example}~
\begin{verbatim}
\begin{frame}{Example AF}
\begin{af}
    \argument<1>{a}                 % Argument visible on slide 1
    \argument<1,2>[right=of a1]{b}  % Argument visible on slides 2 and 3
    \argument<2-4>[right=of a2]{c}  % Argument visible on slides 2 to 4

    \attack<2->{a2}{a3}             % Attack is visible on slide 2 
                                    % and any following slides
\end{af}
\end{frame}
\end{verbatim}
\end{example}

In accordance with the \textsf{beamer} cover types, the \argumentation package provides the two cover modes: \textsf{transparent} and \textsf{invisible}, activated by giving the option \textsf{covered=transparent} or \textsf{covered=invisible} to the \texttt{af} environment (default is \textsf{transparent}).
In \textsf{transparent} mode arguments and attacks that are inactive on a slide are rendered in gray, and in \textsf{invisible} mode they are completely invisible on inactive slides.

\iffalse
\subsubsection{Experimental Features}
\textbf{Warning: The following features are experimental and the performance is not necessarily optimized yet.}
If you want to reuse (parts of) previously created argumentation frameworks in some form, the \argumentation package provides some useful commands that can be enabled via the package option $\textsf{beamer}{=}\textsf{true}$.
While primarily intended for the use inside the $\textsf{beamer}$ document class when creating presentations, the commands also work in any other document class.
Each command required the label of some argumentation framework and a list of argument IDs of that framework and then creates a copy of that framework, with some changes depending on the command.\\

\noindent\cs{aflabeling \{\mopt{af-label}\} \{\mopt{argument list}\}}

    \noindent
    Applies the style parameter \textsf{accepted} to all arguments in \mopt{argument list}, the parameter \textsf{rejected} to those attacked by arguments in \mopt{argument list} and \textsf{undecided} to all other arguments.\\

\noindent\cs{afextension \{\mopt{af-label}\} \{\mopt{argument list}\}}

    \noindent
    Applies the style parameter \textsf{accepted} to all arguments in \mopt{argument list}.\\

\noindent\cs{afreduct \{\mopt{af-label}\} \{\mopt{argument list}\}}

    \noindent
    Applies the style parameter \textsf{inactive} to all arguments in \mopt{argument list} and those attacked by them.
    All attacks involving at least one such argument also receive the parameter \textsf{inactive}.\\

\noindent\cs{afrestriction \{\mopt{af-label}\} \{\mopt{argument list}\}}

    \noindent
    Applies the style parameter \textsf{invisible} to all arguments \emph{not} in \mopt{argument list}.
    All attacks involving at least one such argument also receive the parameter \textsf{invisible}.\\


See Figure~\ref{fig:beamer} for some examples.

\begin{figure}[!ht]
    %\setafstyle{small}
    \setargumentstyle{argument thick}
    %\centering
    \begin{subfigure}{0.48\textwidth}
        \centering
        \begin{af*}[namestyle=monospace]
            \aflabeling{af:example}{a1}
        \end{af*}
        \caption{Result of \cs[0cm]{aflabeling\{af:example\}\{a1\}}.}
    \end{subfigure}
    \begin{subfigure}{0.48\textwidth}
        \centering
        \begin{af*}[namestyle=monospace]
            \afextension{af:example}{a1,a3}
        \end{af*}
        \caption{Result of \cs[0cm]{afextension\{af:example\}\{a1,a3\}}.}
    \end{subfigure}

    \par\bigskip
    
    \begin{subfigure}{0.48\textwidth}
        \centering
        \begin{af*}[namestyle=monospace]
            \afreduct{af:example}{a2}
        \end{af*}
        \caption{Result of \cs[0cm]{afreduct\{af:example\}\{a2\}}.}
    \end{subfigure}
    \begin{subfigure}{0.48\textwidth}
        \centering
        \begin{af*}[namestyle=monospace]
            \afrestriction{af:example}{a3,a4}
        \end{af*}
        \caption{Result of \cs[0cm]{afrestriction\{af:example\}\{a3,a4\}}.}
    \end{subfigure}
    \caption{Example usage of the four commands provided by the \textsf{beamer} package option.}
    \label{fig:beamer}
\end{figure}

\fi

%%%%%%%%%%%%%%%%%%%%%%%%%%%%%%%%%%%%%%%%%%%%%%%%%%%%%%%%%%%%%%%%%%%%%%%%%%%%%%%%%%%%%%%%%%%
\subsection{Other Commands}
%%%%%%%%%%%%%%%%%%%%%%%%%%%%%%%%%%%%%%%%%%%%%%%%%%%%%%%%%%%%%%%%%%%%%%%%%%%%%%%%%%%%%%%%%%%

The \argumentation package also provides some additional features.
The following command can be used to create a text annotation next to an argument node in the argumentation framework.
The annotation we be placed above the argument node, to adjust its position you should use \texttt{xshift} and \texttt{yshift}.\\

\noindent\cs{annotation [\oopt{options}] \{\mopt{argId}\} \{\mopt{value}\}}
\begin{description}
    \item[\oopt{options}] Optional \tikzname style parameters.
    \item[\mopt{argId}] Identifier of the argument.
    \item[\mopt{value}] The annotation text.\\
\end{description}

The command \cs[0cm]{afname} can be used to create a simple text node inside the AF.
Mainly intended to add the name of the AF into the picture, you can generally put any text there.
The command behaves essentially exactly like the \cs[0cm]{argument} command.\\

\noindent\cs{afname [\oopt{options}] (\popt{id}) \{\mopt{name}\} at (\popt{posX}, \popt{posY})}
\begin{description}
    \item[\oopt{options}] (optional) a list of \tikzname style parameters and/or relative positioning parameters.
    \item[\popt{id}] (optional) Identifier of the text node. If omitted the identifier will be \textsf{caption}.
    \item[\mopt{text}] Text to be displayed.
    \item[\popt{posX},\popt{posY}] (optional) the coordinates for placement. Omit if using relative positioning.
\end{description}

If you want to define your own style for arguments, attacks or supports, you may use one of the following commands to override the package-wide settings to your liking.
For that you may also reuse some of the pre-defined parameters of the \argumentation package (see Section~\ref{sec:parameters}).\\

\noindent\cs{setargumentstyle \{\mopt{style parameters}\}}

\noindent\cs{setatackstyle \{\mopt{style parameters}\}}

\noindent\cs{setsupportstyle \{\mopt{style parameters}\}}\\

Similarly, you can also use the following command to override the default settings of the \texttt{af} environment, e.\,g., the \texttt{node distance}.\\

\noindent\cs{setafstyle \{\mopt{style parameters}\}}\\

Finally, when using the $\textsf{argumentstyle}{=}\emph{colored}$ package option, you may use the following command to set the color scheme.\\

\noindent\cs{setargumentcolorscheme \{\mopt{outer color}\} \{\mopt{inner color}\}}

\newpage
\begin{example}~

    \begin{minipage}{0.22\textwidth}
        \centering
        \setargumentstyle{argument thick,text=blue}
        \setafstyle{node distance=0.75cm}
        \begin{af}
            \label{af:8}
            \argument{a}
            \argument[right=of a1]{b}
            \argument[below=of a1]{c}
            \argument[right=of a3]{d}

            \annotation[yshift=-0.2cm]{a4}{$\top$}
            \afname{$F_{\ref{af:8}}$} at (-1,-0.75)

            \attack{a1}{a2}
            \dualattack{a1}{a3}
            \attack{a4}{a3}
        \end{af}
    \end{minipage}%
    \begin{minipage}{0.7\textwidth}
        \begin{verbatim}
            \setargumentstyle{argument thick,text=blue}
            \setafstyle{node distance=0.75cm}
            \begin{af}\label{af:8}
                ...
        
                \annotation[yshift=-0.2cm]{a4}{$\top$}
                \afname{$F_{\ref{af:8}}$} at (-1,-0.75)
            \end{af}
        \end{verbatim}
    \end{minipage}
\end{example}

%%%%%%%%%%%%%%%%%%%%%%%%%%%%%%%%%%%%%%%%%%%%%%%%%%%%%%%%%%%%%%%%%%%%%%%%%%%%%%%%%%%%%%%%%%%
\subsubsection{Argumentation Macros}\label{sec:macros}
%%%%%%%%%%%%%%%%%%%%%%%%%%%%%%%%%%%%%%%%%%%%%%%%%%%%%%%%%%%%%%%%%%%%%%%%%%%%%%%%%%%%%%%%%%%
To facilitate referencing argumentation frameworks and working with them in general, the \argumentation package provides some additional macros that can be enabled with the package option \textsf{macros=true}.
Most importantly, there is the macro \verb|\afref{|\sopt{label}\verb|}| which works like the \verb|\ref| command but adds the reference number directly into the index of the \verb|\AF| symbol.
You may redefine any of the first four commands if you prefer a different naming scheme for AFs.

\begin{table}[ht]
    \centering
    \begin{tabular}{lll}
        \verb|\AF|&& \AF \\
        \verb|\arguments|&& \arguments\\
        \verb|\attacks|&&\attacks\\
        \verb|\AFcomplete|&&\AFcomplete\\
        \verb|\afref{af:example}|&&\afref{af:example}\\
        \verb|\fullafref{af:example}|&\qquad\qquad\qquad&\fullafref{af:example}
    \end{tabular}
    \caption{Macros provided by the package option \textsf{macros=true} and their respective output.}
    \label{tab:macros}
\end{table}

With the following command you can define a new type of argumentation environment, which will have its own separate counter and accompanying \verb|\ref| command, analogously to the \texttt{af} environment.\\

\noindent\cs{newafenvironment \{\mopt{environment name}\} \{\mopt{macro name}\}}
\begin{description}
    \item[\mopt{environment name}] name of the environment to be defined.
    \item[\mopt{macro name}] name of the accompanying macro for the type of argumentation framework.\\
\end{description}

For instance, \verb|\newafenvironment{adf}{\ADF}| creates a new environment called \texttt{adf} and the command \verb|\adfref{ }|, which indexes the symbol you have defined via \verb|\ADF|.

\newpage
%%%%%%%%%%%%%%%%%%%%%%%%%%%%%%%%%%%%%%%%%%%%%%%%%%%%%%%%%%%%%%%%%%%%%%%%%%%%%%%%%%%%%%%%%%%%%%%%%%%%%%%%%%%
\section{Package Options}\label{sec:options}
%%%%%%%%%%%%%%%%%%%%%%%%%%%%%%%%%%%%%%%%%%%%%%%%%%%%%%%%%%%%%%%%%%%%%%%%%%%%%%%%%%%%%%%%%%%%%%%%%%%%%%%%%%%
The \argumentation package comes with some package options to customize the appearance of the created argumentation frameworks as well as some additional features.
All style package options can both be set globally when importing the package and also locally for each \texttt{af} environment.
To import the \argumentation package, use the following command in the preamble of your \LaTeX-document:\\

\noindent\cs[0cm]{usepackage[\oopt{options}]\{argumentation\}}\\

The following package options are currently available:

\begin{description}
    \item[argumentstyle] (default \texttt{standard}) Globally sets the appearance of the argument nodes.
    The \argumentation package provides five options: \texttt{standard}, \texttt{large}, \texttt{thick}, \texttt{gray} and \texttt{colored}.
    Detailed descriptions of these options can be found below.
    
    \item[attackstyle] (default \texttt{standard}) Globally sets the appearance of the attack edges.
    The package comes with three available options: \texttt{standard}, \texttt{large} and \texttt{modern}.
    Detailed descriptions of these options can be found below.
    
    \item[supportstyle] (default \texttt{standard}) Globally sets the appearance of the support edges.
    The package comes with three available options: \texttt{standard}, \texttt{dashed} and \texttt{double}.
    Detailed descriptions of these options can be found below.
    
    \item[namestyle] (default \texttt{none}) Sets the text formatting applied to the argument names in the document.
    The package comes with five available options: \texttt{none}, \texttt{math}, \texttt{bold}, \texttt{monospace} and \texttt{monoemph}.
    Detailed descriptions of these options can be found below.
    
    \item[indexing]  (default \texttt{numeric}) Enables or disables automatic generation of \tikzname node-IDs for the created arguments.
    The available options are: \texttt{none}, \texttt{numeric} and \texttt{alphabetic}.
    Under the default numeric indexing the generated argument IDs are of the form $a1, a2, \dots$.
    With alphabetic indexing the IDs will simply be letters: $a,b,\dots$.
    If \texttt{none} is selected, no IDs will be generated and you are required to provide them for each argument via the parameter \texttt{(\popt{id})} of the \verb|\argument| command.
    
    \item[macros] Boolean (default \texttt{false}) When enabled provides additional macros for naming and referencing argumentation frameworks (see Table~\ref{tab:macros}).
    \item[beamer] Boolean (default \texttt{false}) When enabled, provides the commands for recreating argumentations frameworks described in Section~\ref{sec:beamer}.
\end{description}

In the following we give an overview of the different options for the style parameters that can be used to customise the created argumentation frameworks.
For the exact definitions of these parameters, refer to Section~\ref{sec:parameters}.

\newpage
\paragraph{\sffamily argumentstyle=\mopt{option}}
\begin{align*}
    \mathsf{standard} &\quad \text{Circular argument node with normal size argument name.}\\
    \mathsf{large} &\quad \text{Larger font of the argument name.}\\
    \mathsf{thick} &\quad \text{Thick black outline and normal size argument name.}\\
    \mathsf{gray} &\quad \text{Thick gray outline, light gray background.}\\
    \mathsf{colored} &\quad \text{Thick blue outline, light blue background.}
\end{align*}

\begin{figure}[!h]
    \begin{subfigure}{0.32\textwidth}
        \centering
        \begin{af}[argumentstyle=standard]
            \argument{a}
            \argument[right=of a1]{b}
    
            \attack[]{a1}{a2}
            \label{af:test}
        \end{af}
        \caption{\textsf{argumentstyle}=\textit{standard}}
        \label{fig:argumentstyle_standard}
    \end{subfigure}
    \hfill
    \begin{subfigure}{0.3\textwidth}
        \centering
        \begin{af}[argumentstyle=large]
            \argument{a}
            \argument[right=of a1]{b}
    
            \attack[]{a1}{a2}
        \end{af}
        \caption{\textsf{argumentstyle}=\textit{large}}
        \label{fig:argumentstyle_large}
    \end{subfigure}
    \hfill
    \begin{subfigure}{0.3\textwidth}
        \centering
        \begin{af}[argumentstyle=thick]
            \argument{a}
            \argument[right=of a1]{b}
    
            \attack[]{a1}{a2}
        \end{af}
        \caption{\textsf{argumentstyle}=\textit{thick}}
        \label{fig:argumentstyle_thick}
    \end{subfigure}

    \par\bigskip

    \begin{subfigure}{0.49\textwidth}
        \centering
        \begin{af}[argumentstyle=gray]
            \argument{a}
            \argument[right=of a1]{b}
    
            \attack[]{a1}{a2}
        \end{af}
        \caption{\textsf{argumentstyle}=\textit{gray}}
        \label{fig:argumentstyle_gray}
    \end{subfigure}
    \hfill
    \begin{subfigure}{0.49\textwidth}
        \centering
        \begin{af}[argumentstyle=colored]
            \argument{a}
            \argument[right=of a1]{b}
    
            \attack[]{a1}{a2}
        \end{af}
        \caption{\textsf{argumentstyle}=\textit{colored}}
        \label{fig:argumentstyle_colored}
    \end{subfigure}
    
    \caption{Available options for \textsf{argumentstyle}.}
    \label{fig:argumentstyle}
\end{figure}


\paragraph{\sffamily attackstyle=\mopt{option}}
    
\begin{align*}
    \mathsf{standard} &\quad \text{Standard 'stealth' \tikzname arrow tip.}\\
    \mathsf{large} &\quad \text{Arrow tip is larger and sharper.}\\
    \mathsf{modern} &\quad \text{\tikzname ModernCS arrow tip.}
\end{align*}

\begin{figure}[!h]
    \begin{subfigure}{0.32\textwidth}
        \centering
        \begin{af}[attackstyle=standard]
            \argument{a}
            \argument[right=of a1]{b}
    
            \attack[]{a1}{a2}
        \end{af}
        \caption{\textsf{attackstyle}=\textit{standard}}
        \label{fig:attackstyle_standard}
    \end{subfigure}
    \hfill
    \begin{subfigure}{0.3\textwidth}
        \centering
        \begin{af}[attackstyle=large]
            \argument{a}
            \argument[right=of a1]{b}
    
            \attack[]{a1}{a2}
        \end{af}
        \caption{\textsf{attackstyle}=\textit{large}}
        \label{fig:attackstyle_large}
    \end{subfigure}
    \hfill
    \begin{subfigure}{0.3\textwidth}
        \centering
        \begin{af}[attackstyle=modern]
            \argument{a}
            \argument[right=of a1]{b}
    
            \attack[]{a1}{a2}
        \end{af}
        \caption{\textsf{attackstyle}=\textit{modern}}
        \label{fig:attackstyle_thick}
    \end{subfigure}
    \caption{Available options for \textsf{attackstyle}.}
    \label{fig:attackstyle}
\end{figure}


\paragraph{\sffamily supportstyle=\mopt{option}}
    
\begin{align*}
    \mathsf{standard} &\quad \text{Same tip as attack arrow, perpendicular mark on arrow line.}\\
    \mathsf{dashed} &\quad \text{Dashed arrow line and same tip as attack arrow.}\\
    \mathsf{double} &\quad \text{Double arrow line and large flat tip.}
\end{align*}

\begin{figure}[!h]
    \begin{subfigure}{0.32\textwidth}
        \centering
        \begin{af}[supportstyle=standard]
            \argument{a}
            \argument[right=of a1]{b}
    
            \support[]{a1}{a2}
        \end{af}
        \caption{\textsf{supportstyle}=\textit{standard}}
        \label{fig:supportstyle_standard}
    \end{subfigure}
    \hfill
    \begin{subfigure}{0.3\textwidth}
        \centering
        \begin{af}[supportstyle=dashed]
            \argument{a}
            \argument[right=of a1]{b}
    
            \support[]{a1}{a2}
        \end{af}
        \caption{\textsf{supportstyle}=\textit{dashed}}
        \label{fig:supportstyle_dashed}
    \end{subfigure}
    \hfill
    \begin{subfigure}{0.3\textwidth}
        \centering
        \begin{af}[supportstyle=double]
            \argument{a}
            \argument[right=of a1]{b}
    
            \support[]{a1}{a2}
        \end{af}
        \caption{\textsf{supportstyle}=\textit{double}}
        \label{fig:supportstyle_double}
    \end{subfigure}
    \caption{Available options for \textsf{supportstyle}. Note that for \textit{standard} and \textit{dashed} the arrow tip of the selected \textsf{attackstyle} will be used.}
    \label{fig:supportstyle}
\end{figure}


\paragraph{\sffamily namestyle=\mopt{option}}
    
\begin{align*}
    \mathsf{none} &\quad \text{No effect applied to argument name.}\\
    \mathsf{math} &\quad \text{The argument name is rendered as $math$ text.}\\
    &\quad\quad \text{(name must be given without mathmode).}\\
    \mathsf{bold} &\quad \text{The argument name is rendered in $\boldsymbol{bold}$.}\\
    &\quad\quad \text{(name must be given without mathmode).}\\
    \mathsf{monospace} &\quad \text{The argument name is rendered in \texttt{monospace} font.}\\
    &\quad\quad \text{(name must be given without mathmode).}\\
    \mathsf{monoemph} &\quad \text{The argument name is rendered as {\ttfamily\emph{name}}.}
\end{align*}

\begin{figure}[!ht]
    \begin{subfigure}{0.32\textwidth}
        \centering
        \begin{af}[namestyle=none]
            \argument{a}
            \argument[right=of a1]{b}
    
            \attack[]{a1}{a2}
        \end{af}
        \caption{\textsf{namestyle}=\textit{none}}
        \label{fig:namestyle_none}
    \end{subfigure}
    \hfill
    \begin{subfigure}{0.32\textwidth}
        \centering
        \begin{af}[namestyle=math]
            \argument{a}
            \argument[right=of a1]{b}
    
            \attack[]{a1}{a2}
        \end{af}
        \caption{\textsf{namestyle}=\textit{math}}
        \label{fig:namestyle_math}
    \end{subfigure}
    \hfill
    \begin{subfigure}{0.32\textwidth}
        \centering
        \begin{af}[namestyle=bold]
            \argument{a}
            \argument[right=of a1]{b}
    
            \attack[]{a1}{a2}
        \end{af}
        \caption{\textsf{namestyle}=\textit{bold}}
        \label{fig:namestyle_bold}
    \end{subfigure}
    \hfill
    \par\bigskip
    \begin{subfigure}{0.49\textwidth}
        \centering
        \begin{af}[namestyle=monospace]
            \argument{a}
            \argument[right=of a1]{b}
    
            \attack[]{a1}{a2}
        \end{af}
        \caption{\textsf{namestyle}=\textit{monospace}}
        \label{fig:namestyle_monospace}
    \end{subfigure}
    \hfill
    \begin{subfigure}{0.49\textwidth}
        \centering
        \begin{af}[namestyle=monoemph]
            \argument{a}
            \argument[right=of a1]{b}
    
            \attack[]{a1}{a2}
        \end{af}
        \caption{\textsf{namestyle}=\textit{monoemph}}
        \label{fig:namestyle_monoemph}
    \end{subfigure}
    \caption{Available options for \textsf{namestyle}. You can of course apply any formatting yourself when using the default \textsf{namestyle}=\textit{none}.}
    \label{fig:namestyle}
\end{figure}

\newpage
%%%%%%%%%%%%%%%%%%%%%%%%%%%%%%%%%%%%%%%%%%%%%%%%%%%%%%%%%%%%%%%%%%%%%%%%%%%%%%%%%%%%%%%%%%%%%%%%%%%%%%%%%%%
\section{Style Parameter Reference}\label{sec:parameters}
%%%%%%%%%%%%%%%%%%%%%%%%%%%%%%%%%%%%%%%%%%%%%%%%%%%%%%%%%%%%%%%%%%%%%%%%%%%%%%%%%%%%%%%%%%%%%%%%%%%%%%%%%%%
For reference, the style parameters provided by this package are listed below.
You may use or redefine them at your own discretion.

\begin{table}[ht]
        \centering
        \ttfamily
        \begin{tabular}{l|l}
            \toprule
            \tikzname-keyword & style parameters \\
            \midrule
            \textsf{argument size} & \emph{contains the currently selected argument size}\\
            \textsf{argument} & \emph{contains the currently selected argument style and size}\\
            \textsf{argument standard} & circle,inner sep=0,outer sep=0,draw=black\\
            \textsf{argument large} & circle,inner sep=0,outer sep=0,draw=black,font=\verb|\|large\\
            \textsf{argument thick} & circle,inner sep=0,outer sep=0,draw=black,line width=0.1em\\
            \textsf{argument gray} & argument thick,fill=gray!30,draw=gray!65,text=black!80\\
            \textsf{argument colored} & argument thick,fill=aigblue!40,draw=aigblue!80,text=black!80\\
            \midrule
            \textsf{attack} & \emph{contains the currently selected attack style}\\
            \textsf{attack standard} & -\{stealth'\}\\
            \textsf{attack large} & -\{Stealth[scale=1.25]\}\\
            \textsf{attack modern} & -\{To[sharp,length=0.65ex,line width=0.05em]\}\\
            \textsf{selfattack} & loop,min distance=0.4em,in=0,out=60,looseness=4.5\\
            \midrule
            \textsf{support} & \emph{contains the currently selected support style}\\
            \textsf{support standard} & attack,postaction=\{decorate,decoration=\{\dots\}\}\\
            \textsf{support dashed} & attack,densely dashed\\
            \textsf{support double} & -\{Classical TikZ Rightarrow\},double\\
            \midrule
            \textsf{inactive} & fill=none,draw=gray!50,text=gray!60\\
            \textsf{incomplete} & densely dashed\\
            \textsf{accepted} & fill=green!40\\
            \textsf{rejected} & fill=red!40\\
            \textsf{undecided} & fill=cyan!40\\
            \textsf{highlight} & fill=aigyellow!60\\
            \textsf{invisible} & draw=none,fill=none,opacity=0.0\\
            \midrule
            \textsf{standard}       & node distance=6.6ex,argument size/.style={minimum size=4.5ex},\\
                                    & attack width/.style={line width=0.05em}\\
            \textsf{small}          & node distance=3.5ex,argument size/.style={minimum size=3.4ex},\\
                                    & attack width/.style={line width=0.045em}\\
            \textsf{tiny}           & node distance=2.3ex,argument size/.style={minimum size=2.6ex,}\\
                                    & attack width/.style={line width=0.03em},font=\verb|\small|\\
            \bottomrule
        \end{tabular}
        \caption{Reference list of \tikzname-style parameters provided by the \argumentation package.}
        \label{tab:styles}
    \end{table}
\newpage
%%%%%%%%%%%%%%%%%%%%%%%%%%%%%%%%%%%%%%%%%%%%%%%%%%%%%%%%%%%%%%%%%%%%%%%%%%%%%%%%%%%%%%%%%%%%%%%%%%%%%%%%%%%
\section{Version History}\label{sec:history}
%%%%%%%%%%%%%%%%%%%%%%%%%%%%%%%%%%%%%%%%%%%%%%%%%%%%%%%%%%%%%%%%%%%%%%%%%%%%%%%%%%%%%%%%%%%%%%%%%%%%%%%%%%%
\subsection*{[v1.6 2025/07/31]}
\begin{itemize}
    \item Added command \verb|\newafenvironment{ }{ }| to dynamically add new \texttt{af} environments.
    \item Updated \verb|\afref{ }| command with better hyperlink.
    \item Removed experimental features.
\end{itemize}

\subsection*{[v1.5 2025/02/09]}
\begin{itemize}
    \item Added support for beamer overlay specifications to \verb|\argument{ }|, \verb|\attack{ }{ }| and related commands.
    \item Added \textsf{covered} parameter for \texttt{af} environment to select between \textit{transparent} and \textit{invisible} rendering of inactive arguments/attacks on slides.
    \item Bugfix related to experimental beamer commands.
\end{itemize}

\subsection*{[v1.4 2024/10/31]}
\begin{itemize}
    \item Added functions \verb|\aflabeling|, \verb|\afextension|, \verb|\afreduct| and \verb|\afrestriction| that recreate (parts of) previously created argumentation frameworks. Can be enabled via the package option \textsf{beamer=true}.
    \item Added internal storage of arguments and attacks of an argumentation framework to enable further computations.
    \item Added environment \verb|af*| for argumentation frameworks that are unlabeled/uncounted.
    \item Added command \verb|\setargumentcolorscheme{ }{ }| to change color scheme of the \textsf{colored} argument style.
    \item Added command \verb|\setafstyle{ }| to set global style options for the AFs.
    \item Added optional parameter (\popt{value}) to \verb|\attack| command to add a label to the attack edge (undocumented for now).
    \item Major revision of the documentation.
    \item Various minor changes to internal functions, naming scheme and comments.
\end{itemize}
\subsection*{[v1.3 2024/09/25]}
\begin{itemize}
    \item Added support for \verb|\label{ }| and \verb|\ref{ }| to \texttt{af} environment.
    \item Added commands \verb|\AF|, \verb|\arguments|, \verb|\attacks| and \verb|\AFcomplete| to facilitate consistent naming of AFs. Have to be loaded with the package option \textsf{macros=true}.
    \item Added commands \verb|\afref{ }| and \verb|\fullafref{ }| to reference AFs.
    \item adjusted scaling of nodes and arrows for larger page sizes.
    \item added new style options for arguments.
    \item Various minor fixes and changes regarding the \textsf{namestyle} package option.
\end{itemize}

\subsection*{[v1.2 2024/06/07]}
\begin{itemize}
    \item Changed Syntax of \verb|\argument| command. The \textit{id} parameter is now given inside parenthesis instead of curly braces and is optional.
    \item Added absolute positioning to \verb|\argument| command, like for \tikzname nodes.
    \item Added package option $\textsf{indexing}$ to toggle automatic generation of identifiers for created argument nodes. Can be set to \textit{none}, or selected between \textit{alphabetic} and \textit{numeric} (default).
    \item All package style options can now also be set locally in the \texttt{af} environment.
    \item Adjusted \verb|\annotatedattack| to require position parameter.
    \item Various minor bugfixes regarding the \textsf{namestyle} package option.
    \item Added new argumentstyle \textsf{large}.
\end{itemize}

\subsection*{[v1.1 2023/12/03]}
\begin{itemize}
    \item Adjusted standard styles.
    \item Added command for creating annotated attacks.
    \item Now only provides one environment, which can be parameterised.
    \item Changed option management to pgfkeys.
    \item Updated and improved documentation.
\end{itemize}

\subsection*{[v1.0 2023/11/05]}
\begin{itemize}
    \item First Version.
\end{itemize}

\end{document}