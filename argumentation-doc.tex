\documentclass{article}

\title{The \texttt{argumentation} Package}
\author{Lars Bengel\\\small lars.bengel@fernuni-hagen.de}
%\date{}

%%%%%%%%% IMPORTS %%%%%%%%%%%%%%%%%%%%%%%%%%%%%%%%%%%%%%%%%%%%%%%%%%%%%%%%%%%%%%%%%%%%%%%%%%
\usepackage{argumentation}
\usepackage{amsmath}

\newtheorem{example}{Example}
%%%%%%%%%%%%%%%%%%%%%%%%%%%%%%%%%%%%%%%%%%%%%%%%%%%%%%%%%%%%%%%%%%%%%%%%%%%%%%%%%%%%%%%%%%%%

\begin{document}
\maketitle

\tableofcontents
\newpage

\section{Example}
\vspace{-0.7cm}
\begin{figure}[ht]
    \centering
    \begin{af}
        \argument{args1}{$a_1$}
        \argument[right=of args1]{args2}{b}
        \argument[right=of args2]{args3}{c}
        \argument[right=of args3]{args4}{d}
        \argument[right=of args4]{args5}{e}
        \argument[below=of args1]{args6}{f}
        \argument[inactive,right=of args6]{args7}{g}
        \argument[inactive,argin,right=of args7]{args8}{h}
        \argument[right=of args8]{args9}{i}
        \argument[right=of args9]{args10}{j}

        \afname[left of=args1,yshift=-0.8cm,xshift=-0.2cm]{cap}{\textbf{F:}}

        \selfattack{args1}
        \dualattack[]{args1}{args6}
        \dualattack[inactive]{args6}{args7}
        
        \attack[inactive]{args8}{args7}
        \attack[inactive]{args7}{args2}
        \attack[]{args3}{args2}    
        \attack[]{args4}{args5}
        \attack[]{args5}{args10}
        \attack[]{args10}{args9}
        \attack[]{args9}{args4}

        \support[]{args4}{args3}
        \support[]{args9}{args3}
    \end{af}
    \caption{An exemplary AF created with the \textsf{argumentation} package.}
    \label{fig:example}
\end{figure}
\vspace{-0.4cm}
\begin{verbatim}
    \usepackage{argumentation}
    \begin{figure}[ht]
        \centering
        \begin{af}
            \argument{args1}{a}
            \argument[right=of args1]{args2}{b}
            \argument[right=of args2]{args3}{c}
            \argument[right=of args3]{args4}{d}
            \argument[right=of args4]{args5}{e}
            \argument[below=of args1]{args6}{f}
            \argument[inactive,right=of args6]{args7}{g}
            \argument[inactive,argin,right=of args7]{args8}{h}
            \argument[right=of args8]{args9}{i}
            \argument[right=of args9]{args10}{j}
    
            \afname[left of=args1,yshift=-0.8cm,xshift=-0.2cm]{cap}{\textbf{F:}}
    
            \selfattack{args1}
            \dualattack[]{args1}{args6}
            \dualattack[inactive]{args6}{args7}
            
            \attack[inactive]{args8}{args7}
            \attack[inactive]{args7}{args2}
            \attack[]{args3}{args2}    
            \attack[]{args4}{args5}
            \attack[]{args5}{args10}
            \attack[]{args10}{args9}
            \attack[]{args9}{args4}

            \support[]{args4}{args3}
            \support[]{args9}{args3}
        \end{af}
        \caption{An exemplary AF created with the \textsf{argumentation} package.}
        \label{fig:example}
    \end{figure}
\end{verbatim}

\section{Documentation for Version 1.0 [2023/11/05]}
In the following, we provide an overview over the functionality of the \textsf{argumentation} package.

\subsection{Package Options}
    Multiple options are provided to customize the look of the argumentation framework.
    The \textsf{namestyle} option accepts three different values
    \begin{align*}
        \mathsf{italics} &\quad \text{(default) The argument name is rendered in \emph{italics}.}\\
        \mathsf{bold} &\quad \text{The argument name is rendered in \textbf{bold}.}\\
        \mathsf{bolditalics} &\quad \text{The argument name is rendered with \textbf{\emph{both}}.}\\
        \mathsf{normal} &\quad \text{The argument name is rendered normally.}
    \end{align*}

    The \textsf{argumentstyle} option controls the style of the argument nodes and accepts two values
        \begin{align*}
            \mathsf{standard} &\quad \text{(default) Standard style for the argument nodes.}\\
            \mathsf{retro} &\quad \text{Alternative style, thicker outline and slightly larger nodes.}
        \end{align*}

    The \textsf{attackstyle} option controls the style of the attack arrows and accepts two values
    \begin{align*}
        \mathsf{standard} &\quad \text{(default) Standard style for the attack arrow tips.}\\
        \mathsf{retro} &\quad \text{Alternative style, arrow tip is larger and sharper.}
    \end{align*}

    The \textsf{supportstyle} option controls the style of the support arrows and accepts three values
    \begin{align*}
        \mathsf{standard} &\quad \text{(default) Standard style for the attack arrow tips.}\\
        \mathsf{dashed} &\quad \text{Dashed arrow line, same tip.}\\
        \mathsf{double} &\quad \text{Double arrow line and large flat tip.}\\
    \end{align*}
    

\subsection{Environments}
The package provides two environments for creating abstract argumentation frameworks and bipolar argumentation frameworks in \LaTeX-documents.

\subsubsection{\textsf{af}-Environment}
    The \textsf{argumentation} package provides the \textsf{af} environment for creating abstract argumentation framework.
    The \textsf{af} environment extends the \textsf{tikzpicture} environment, meaning all \textsf{tikzpicture}-parameters can be used inside the \textsf{af} environment as well.
    The most relevant parameter is \verb|node distance|, which is set to \verb|1cm| per default.
    
\subsubsection{\textsf{miniaf}-Environment}
    The \textsf{miniaf} environment can be used to create argumentation frameworks using less space.
    Especially useful for two-column layout documents.
    It provides essentially the same as the \textsf{af} environment, but with \verb|node distance=0.5cm| and for each node \verb|minimum size=0.5cm|, \verb|font=\small|.


\begin{figure}[ht]
    \centering
    \begin{miniaf}
        \argument{args1}{a}
        \argument[right=of args1]{args2}{b}
        \argument[right=of args2]{args3}{c}
        \argument[right=of args3]{args4}{d}
        \argument[right=of args4]{args5}{e}
        \argument[below=of args1]{args6}{f}
        \argument[inactive,right=of args6]{args7}{g}
        \argument[inactive,argin,right=of args7]{args8}{h}
        \argument[right=of args8]{args9}{i}
        \argument[right=of args9]{args10}{j}

        \afname[left of=args1,yshift=-0.5cm,xshift=-0.2cm]{cap}{\textbf{F:}}

        \selfattack{args1}
        \dualattack[]{args1}{args6}
        \dualattack[inactive]{args6}{args7}
        \attack[inactive]{args8}{args7}

        \attack[inactive]{args7}{args2}
        \attack[]{args3}{args2}

        \support[]{args4}{args3}
        \support[]{args9}{args3}

        \attack[]{args4}{args5}
        \attack[]{args5}{args10}
        \attack[]{args10}{args9}
        \attack[]{args9}{args4}
    \end{miniaf}
    \caption{An exemplary Mini-AF created with the \textsf{miniaf} environment.}
    \label{fig:example_mini}
\end{figure}


\subsection{Arguments}
    Arguments can be created with the following command

    \verb|\argument{id}{name}|

    \noindent
    To create an argument, you must provide a unique identifier \textsf{id} and the \textsf{name} to be displayed in the picture.
    The \textsf{id} of an argument is then referred to when creating attacks as well as for the relative positioning of the other arguments.

    The \textsf{standard} style of an argument is defined with the following parameters, all of which can be overridden if desired.
    \begin{align*}
        \textsf{circle} &\quad \text{the shape of the argument.}\\
        \textsf{minimum~size=0.75cm} &\quad \text{the minimum size of the circle, to ensure consistent}\\
        &\quad \text{argument size.}\\
        \textsf{draw=black} &\quad \text{outline and text color of the argument.}\\
        \textsf{thick} &\quad \text{the outline of the circle is rendered in \textsf{thick} mode.}\\
        \textsf{fill=white} &\quad \text{the background color of the argument.}\\
        \textsf{font=large} &\quad \text{the font size of the argument name.}\\
        \textsf{text~centered} &\quad \text{positioning of the argument name inside the circle.}\\
        \textsf{inner~sep=0} &\quad \text{inner margins of the circle, set to \textsf{0} to optimize space.}
    \end{align*}
    \subsubsection{Relative Positioning}    
    This package supports relative placement of the arguments via the \textsf{tikz}-library \textsf{positioning}.
    The relative positioning information is provided as an optional parameter via
    
    \verb|\argument[dir=of arg_id]{id}{name}|
    
    \noindent
    with \textsf{dir} being one of: \emph{right}, \emph{left}, \emph{below} and \emph{above} and \textsf{arg\_id} being the id of another argument.
    
    Additionally, you can adjust the horizontal/vertical position of an argument via the options \textsf{xshift} and \textsf{yshift}.
    You must also specify the distance in one of the following ways

    \verb|\argument[xshift=5mm]{id}{name}|

    \verb|\argument[xshift=5pt]{id}{name}|

    \verb|\argument[xshift=5ex]{id}{name}|

    \begin{example}~

    \begin{verbatim}
        \begin{af}
            \argument{arg1}{a}
            \argument[right=of arg1]{arg2}{b}
            \argument[right=of arg2, yshift=-10pt]{arg3}{c}
        \end{af}
    \end{verbatim}

    \begin{center}
        \begin{af}
            \argument{arg1}{a}
            \argument[right=of arg1]{arg2}{b}
            \argument[right=of arg2, yshift=-10pt]{arg3}{c}
        \end{af}
    \end{center}
        
    \end{example}

    \subsubsection{Argument Styles}
    Furthermore, you can provide optional parameters to adjust the style of the argument node.
    For that you can use all \textsf{tikz}-style options and additionally the following pre-defined style parameters:
    \begin{align*}
        \mathsf{inactive} &\quad \text{The argument is displayed in grey and with a dotted outline.}\\
        \mathsf{argin} &\quad \text{The argument is displayed with green background color.}\\
        \mathsf{argout} &\quad \text{The argument is displayed with red background color.}\\
        \mathsf{argundec} &\quad \text{The argument is displayed with cyan background color.}
    \end{align*}

    \begin{example}
        
    \end{example}

\subsection{Attacks}
    Attacks between two arguments can be created with the command

    \verb|\attack{arg1}{arg2}|

    \noindent
    where \textsf{arg1} and \textsf{arg2} are the identifiers of two previously defined arguments.
    The \textsf{standard} style for attacks is defined with the \textsf{arrows.meta} library as follows
    \begin{align*}
        \texttt{-{Stealth[scale=1.25]}}
    \end{align*}

\subsubsection{Attack Styles}
    To customize an attack you can provide additional optional parameters:
    \begin{align*}
        \mathsf{inactive} &\quad \text{The attack is displayed in grey and with a dotted line.}\\
        \mathsf{bend~right} &\quad \text{The attack arrow is bent to the right.}\\
        &\quad \text{Can additionally provide the angle, e.\,g., \textsf{bend~right=40}.}\\
        \mathsf{bend~left} &\quad \text{The attack arrow is bent to the left. Can also provide an angle.}\\
    \end{align*}

    Furthermore, all \textsf{tikz} style parameters can be used here as well.

    \begin{example}~
    \begin{verbatim}
        \begin{af}
            \argument{arg1}{a}
            \argument[right=of arg1]{arg2}{b}
            \argument[right=of arg2]{arg3}{c}
            \argument[right=of arg3]{arg4}{d}
    
            \attack{arg1}{arg2}
            \attack[bend right]{arg2}{arg3}
            \attack[bend left=10,inactive]{arg3}{arg4}
        \end{af}    
    \end{verbatim}

    \begin{center}
        \begin{af}
            \argument{arg1}{a}
            \argument[right=of arg1]{arg2}{b}
            \argument[right=of arg2]{arg3}{c}
            \argument[right=of arg3]{arg4}{d}
    
            \attack{arg1}{arg2}
            \attack[bend right]{arg2}{arg3}
            \attack[bend left=10,inactive]{arg3}{arg4}
        \end{af}
    \end{center}
    \end{example}
    

    Additionally, you can create a symmetric attack between two arguments with

    \verb|\dualattack{arg1}{arg2}|

    \noindent
    and a self-attack for an argument with

    \verb|\selfattack{arg1}|

    \noindent
    For both commands, you can use the same optional parameters as for the \verb|\attack| command.

    \begin{example}~
    \begin{verbatim}
        \begin{af}
            \argument{arg1}{a}
            \argument[right=of arg1]{arg2}{b}
    
            \selfattack{arg1}
            \dualattack{arg1}{arg2}
        \end{af}    
    \end{verbatim}

    \begin{center}
        \begin{af}
            \argument{arg1}{a}
            \argument[right=of arg1]{arg2}{b}
    
            \selfattack{arg2}
            \dualattack{arg1}{arg2}
        \end{af}
    \end{center}
    \end{example}
    

\subsection{Supports}
    You can create a support relation between two arguments with the command

    \verb|\support{arg1}{arg2}|

    \noindent
    where \verb|arg1| and \verb|arg2| are the identifiers of two previously defined arguments.
    The support arrow use the same tip as the attack arrows, but have a perpendicular mark to distinguish them from attacks.
    Supports can be customized in the same way as attacks.

    \begin{example}~
    \begin{verbatim}
        \begin{af}
            \argument{arg1}{a}
            \argument[right=of arg1]{arg2}{b}
            \argument[right=of arg2]{arg3}{c}
    
            \support{arg1}{arg2}
            \support[bend right]{arg2}{arg3}
        \end{af}    
    \end{verbatim}

    \begin{center}
        \begin{af}
            \argument{arg1}{a}
            \argument[right=of arg1]{arg2}{b}
            \argument[right=of arg2]{arg3}{c}
    
            \support{arg1}{arg2}
            \support[bend right]{arg2}{arg3}
        \end{af}    
    \end{center}
    \end{example}


\subsection{Further Commands}
    If you want to display an identifier for your argumentation framework in the picture, you can use the command

    \verb|\afname{id}{name}|

    \noindent
    where \verb|id| is an identifier for the created node and \verb|name| is the text displayed in the picture.
    Additionally, positioning information can be provided via the optional parameters.

    \begin{example}~
    \begin{verbatim}
        \begin{af}
            \argument{arg1}{a}
            \argument[right=of arg1]{arg2}{b}
            \afname[left=of arg1]{caption}{$F:$}
    
            \attack{arg1}{arg2}
        \end{af}    
    \end{verbatim}

    \begin{center}
        \begin{af}
            \argument{arg1}{a}
            \argument[right=of arg1]{arg2}{b}
            \argument[right=of arg2]{arg3}{c}
            \afname[left=of arg1]{caption}{$F:$}
    
            \support{arg1}{arg2}
        \end{af}    
    \end{center}
    \end{example}

\end{document}
